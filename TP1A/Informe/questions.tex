
\subsection{Punto 1}

\textbf{Enunciado}: \textsl{Analizar que función cumple y como opera el subcircuito compuesto por $R_{12}$ a $R_{17}$, $C_{16}$ y $U_{3}A$. Luego incluir $R_{S}$. ¿Qué características tiene éste subcircuito, por ejemplo, su transferencia, su ancho de banda, su dependencia de las especificaciones del amplificador operacional $TL082$, de sus fuentes de alimentación, de la temperatura, de la tolerancia y tecnología de los resistores con los que se lo implemente, etc.}

\vspace{1.5cm}

El detalle del cálculo del punto de reposo se realiza en la sección~\sectref{qpoint}, en la figura~\figref{fig:fig_calculated_qpoint} se muestra lo calculado.


\begin{figure}[H] %htb
\begin{center}
\includegraphics[width=0.9 \textwidth, angle=0]{./img/qpoint/polarizacion_calculada.png}
\caption{\label{fig:fig_calculated_qpoint}\footnotesize{Punto de reposo del circuito.}}
\end{center}
\end{figure}


\vfill

\clearpage


\subsection{Punto 2}

\textbf{Enunciado}: \textsl{Analizar qué función cumple y como opera el subcircuito compuesto por $R_{18}$ a $R_{19}$, $C_{15}$ y $U_{3}B$. ¿Qué características tiene éste subcircuito, por ejemplo, su transferencia, su ancho de banda, su dependencia de las especificaciones del amplificador operacional $TL082$, de sus fuentes de alimentación, de la temperatura, de la tolerancia y tecnología de los resistores con lo que se lo implemente, etc. ($R_{18}$ puede variarse desde $0 \si[per-mode=symbol]{\ohm}$ a $18 \si[per-mode=symbol]{\kilo\ohm}$).}

\vspace{1cm}

Para calcular la ganancia de lazo, primero analizamos la realimentación del circuito, observando el circuito de señal a frecuencias media, figura~\figref{fig:fig_scheme_signal_circuit}, se puede observar que la realimentación es \mbox{\textbf{serie-paralelo} (series-shunt)}, es decir se muestrea y se suma tensión.


\begin{figure}[H] %htb
\begin{center}
\includegraphics[width=0.75 \textwidth, angle=0]{./img/desarrollo/2_lazo_cerrado.png}
\caption{\label{fig:fig_scheme_signal_circuit}\footnotesize{Circuito de señal esquematizado.}}
\end{center}
\end{figure}

Para poder analizar la ganancia de lazo, es necesaria la ganancia del camino directo, para esto aplicamos \textbf{parámetros híbridos h} al realimentador del circuito, y lo reorganizamos para llevarlo a la forma ideal, lo que se obtiene se muestra en figura~\figref{fig:fig_scheme_signal_circuit_h_parameters}.


\begin{figure}[H] %htb
\begin{center}
\includegraphics[width=0.75 \textwidth, angle=0]{./img/desarrollo/2-lazoCerradoIdeal.png}
\caption{\label{fig:fig_scheme_signal_circuit_h_parameters}\footnotesize{Circuito de señal esquematizado aplicando parámetros h al realimentador.}}
\end{center}
\end{figure}

\clearpage

\label{calculation_of_f}

\begin{sloppypar}
Donde el parámetro $h_{11} = R_{410} \parallelresistors R_{411} $ queda como resistor de emisor de $Q_{402}$ y $h_{22} = R_{410} + R_{411} $ queda en paralelo con la carga, se desprecia el efecto de $h_{21}$, que es el efecto de la entrada en la salida a través de la realimentación.\\

Se tiene además que $f = h_{12} = \frac{R_{410}}{R_{410} + R_{411}} = \frac{120 \si[per-mode=symbol]{\ohm}}{120 \si[per-mode=symbol]{\ohm} + 3.3 \si[per-mode=symbol]{\kilo\ohm}} \approx 0.035  $
\end{sloppypar}

\subsubsection{Ganancia a lazo abierto del amplificador (\quotemarks{$a$})}
\label{stages_gain}

Para calcular la ganancia del camino directo, ganancia a lazo abierto del amplificador (\textbf{\quotemarks{a}}), se desactiva la realimentación haciendo $f = 0$, el amplificador resultante es la cascada de los dos emisores comunes y la etapa de salida, incluyendo el efecto de carga del realimentador. Todos los cálculos se realizan por inspección, pero teniendo cuidado de usar aproximaciones solo cuando estas sean válidas.\\

La ganancia total será el producto de las ganancias de las tres etapas y no consideramos atenuación en la entrada por calcular suponiendo un generador ideal:\\

\begin{equation}
a = A_{V_{1}} \cdot A_{V_{2}} \cdot A_{V_{3}}
\end{equation}\\





\begin{sloppypar}
Para $A_{V_{1}}$, se cumple que $\beta \gg 1 \; \land \; r_{o} \gg R_{E} \; \land \; gm \cdot r_{o} \gg 1$, podemos usar entonces la expresión aproximada para $A_{V}$: 

$A_{V} \approx -\frac{gm}{1 + gm \cdot R_{E}} \cdot R_{ca}$,  queda entonces:
\end{sloppypar}

\begin{equation}
A_{V_{1}} -\approx \frac{gm_{402}}{1 + gm_{402} \cdot \left( R_{410} \parallelresistors R_{411}\right) } \cdot R_{ca_{1}} = -\frac{gm_{402}}{1 + gm_{402} \cdot \left( R_{410} \parallelresistors R_{411}\right) } \cdot \left( R_{408} \parallelresistors r_{\pi_{404}} \right) \approx -1.8
\end{equation}\\


\begin{sloppypar}
Para $A_{V_{2}}$, tenemos la ganancia de un emisor común: 
\end{sloppypar}


\begin{equation}
A_{V_{2}} = -gm_{404} \cdot R_{ca_{2}} = -gm_{404} \cdot \left(  r{o_{404}} \parallelresistors R_{bt} \parallelresistors R_{i_{3}}  \right)
\end{equation}\\


\begin{sloppypar}
En la figura~\figref{fig:fig_scheme_signal_circuit_h_parameters}, $R_{i_{3}}$ representa la resistencia de entrada de la etapa de salida, y $R_{bt}$ la resistencia que el circuito \mbox{\textbf{bootstrap}} presenta a la segunda etapa, esta última se obtiene en forma aproximada por reflexión de \mbox{\textbf{Miller}} como $R_{bt} = \frac{1}{1 - A_{V_{3}}} \cdot R_{414} $.
La $R_{i_{3}}$ es un tanto mas difícil de calcular, ya que la etapa de salida no funciona en \textbf{clase A}, pero dado que es \textbf{clase AB}, se puede hacer una aproximación suponiendo que se trata de un seguidor por emisor construido alrededor de un \textbf{par Darlington}, en este caso la etapa ni siquiera es completamente simétrica, ya que el papel de transistor \textbf{PNP}, lo cumple un \textbf{par Sziklai}, pero como aproximación para el cálculo manual no es mucho mas lo que se puede hacer, podemos decir que el $\beta$ efectivo del segundo transistor se ve disminuido un poco por la resistencia conectada en su base, teniendo en cuenta esto, aproximamos la resistencia de entrada de la etapa de potencia, como:

\begin{equation*}
R_{i_{3}} = \beta_{405} \cdot \beta_{405_{eff}} \cdot R_{L} \approx \beta_{405} \cdot gm_{407} \cdot \left( r_{\pi_{407}} \parallelresistors R_{419} \right) \cdot R_{L} = 35.56 \si[per-mode=symbol]{\kilo\ohm}
\end{equation*}

Para la ganancia de tensión de la etapa de salida, nuevamente por lo anteriormente dicho, es difícil calcular manualmente un valor, sin embargo por tratarse aproximadamente de un seguidor formado por pares compuestos, asumimos un valor de $A_{V_{3}} = 0.99$, se obtiene valor muy cercano para el \textbf{par Sziklai} realizando todos los cálculos, no tanto para el \textbf{par Darlington}.\\

Las simulaciones nos dirán que tan cercanas a la realidad son estas suposiciones.

\end{sloppypar}


\clearpage


Tenemos entonces:

\begin{equation*}
R_{bt} \approx \frac{1}{1 - A_{V_{3}}} \cdot R_{414} \approx \frac{1}{1 - 0.99} \cdot 2.2 \si[per-mode=symbol]{\kilo\ohm} = 220 \si[per-mode=symbol]{\kilo\ohm}
\end{equation*}


Nos queda entonces:

\label{calculation_of_a}

\begin{equation}
A_{V_{2}} = -gm_{404} \cdot R_{ca_{2}} = -gm_{404} \cdot \left(  r{o_{404}} \parallelresistors R_{bt} \parallelresistors R_{i_{3}}  \right) = -360 \si[per-mode=symbol]{\milli\ampere\per\volt} \cdot \left( 15.9 \si[per-mode=symbol]{\kilo\ohm} \parallelresistors 220 \si[per-mode=symbol]{\kilo\ohm} \parallelresistors 35.56 \si[per-mode=symbol]{\kilo\ohm}  \right) \approx -3770
\end{equation}\\

Con esto, tenemos:

\begin{equation}
a = A_{V_{1}} \cdot A_{V_{2}} \cdot A_{V_{3}} = -1.8 \cdot -3770 \cdot 0.99 \approx 6718
\end{equation}\\


\label{loop_gain}

Finalmente para la ganancia de lazo del amplificador tenemos:

\begin{equation}
\boxed {T = a \cdot f = 6718 \cdot 0.035 \approx 221 }
\end{equation}



\vfill

\clearpage


\subsection{Punto 3}

\textbf{Enunciado}: \textsl{Analizar qué función cumple y cómo opera el subcircuito compuesto por $R_{20}$ a $R_{23}$ y $Q_{7}$-$Q_{9}$-$Q_{10}$-$Q_{11}$. ¿Qué características tiene éste subcircuito?.}

\vspace{1.5cm}

La ganancia global del amplificador, por tratarse de un circuito realimentado, tomando los valores calculados anteriormente para $a$ y $f$ en la sección~\sectref{calculation_of_f}, será:

\begin{equation}
\boxed{ A = \frac{a}{1 + a \cdot f} = \frac{6718}{1 + 6718 \cdot 0.035} \approx 28.5 }
\end{equation}

El valor es prácticamente igual a $\frac{1}{f} \approx 28.6$, dado que se verifica muy bien la condición $a \gg 1$.


\vfill

\clearpage


\subsection{Punto 4}

\textbf{Enunciado}: \textsl{Analizar el subcircuito que proporciona la tensión de referencia. ¿Cómo funciona y qué características tiene? Por ejemplo: hallar por cálculo y por simulación el valor de la tensión de referencia y su dependencia de la variación de la tensión de entrada $V_{1}$, de la temperatura ambiente y de la corriente que pueda entregar éste subcircuito a otros subcircuitos que alimente. Consultar las hojas de datos de todos sus componentes, en especial el $TL431$.}

\vspace{1.5cm}
\label{max_pot}

Para obtener la potencia máxima, debemos suponer algo, en una primera aproximación si suponemos que los transistores son ideales y no hay caídas extra en la etapa de salida, podemos decir que la tensión de pico máxima corresponderá a las tensión de alimentación, que es simétrica, con esta suposición obtenemos:


\begin{equation} \label{eq:1}
P_{max} = \frac{{\hat{V}_{L{max}}}^2}{2 \cdot R_{L}} = \frac{{V_{CC}}^2}{2 \cdot R_{L}} = \frac{30 \si[per-mode=symbol]{\volt}}{2 \cdot 8 \si[per-mode=symbol]{\ohm}} \approx 56.3 \si[per-mode=symbol]{\watt}
\end{equation}

Por supuesto la asunción es falsa en cualquier caso, una estimación mas cercana a la realidad sería utilizar la tensión máxima de salida, que se obtiene de suponer que los transistores de llevan al borde de la operación en modo activo directo, este cálculo se realiza en la sección~\sectref{punto8}, usando ese valor de tensión máxima, obtenemos:

\begin{equation} \label{eq:1}
\boxed{ P_{max} = \frac{{\hat{V}_{L{max}}}^2}{2 \cdot R_{L}} = \frac{26.6 \si[per-mode=symbol]{\volt}}{2 \cdot 8 \si[per-mode=symbol]{\ohm}} \approx 44.2 \si[per-mode=symbol]{\watt} }
\end{equation}



\vfill

\clearpage


\subsection{Punto 5}

\textbf{Enunciado}: \textsl{Analizar el subcircuito compuesto por $Q_{4}$ y $Q_{5}$. Por ejemplo: con que nombre es conocida su topología, comprobar si es una topología que emplea realimentación, qué características funcionales tiene este subcircuito, que valores de impedancia presente a los otros circuitos que alimente, cual es la transferencia de este subcircuito (variable de salida / variable de entrada), cuál es su ancho de banda, etc.}


\vspace{1.5cm}



El circuito formado por los dos transistores, $Q_{4}$ y $Q_{5}$, se trata de un par compuesto Sziklai. En la sección~\sectref{section:sziklai} hacemos un análisis del mismo.\\




\subsection{Punto 6}

\textbf{Enunciado}: \textsl{¿Cuál es el rango de la tensión de salida de la fuente considerando que $R_{9}$ puede variar desde $0 \si[per-mode=symbol]{\ohm}$ a $90 \si[per-mode=symbol]{\kilo\ohm}$? (Tomar $R_{L} = 1 \si[per-mode=symbol]{\mega\ohm}$).}

\vspace{1.5cm}


En la figura~\figref{fig:fig_p6_output_voltage} se muestra el gráfico de la tensión de salida en modo de regulación de tensión en función de la resistencia del resistor $R_{9}$, el gráfico se obtuvo realizando una simulación paramétrica con $R_{L} = 1 \si[per-mode=symbol]{\mega\ohm}$, con el comando \textbf{SPICE} \textit{.step}, y luego se exportó el resultado y se graficó en \textbf{MATLAB}. En el gráfico se puede apreciar que, como se espera según lo calculado, el crecimiento es lineal con $R_{9}$, entre valores muy cercanos a los nominales de $1 \si[per-mode=symbol]{\volt}$ y $10 \si[per-mode=symbol]{\volt}$.




\vfill

\clearpage

\begin{figure}[H] %htb
\begin{center}
\includegraphics[width=1.2 \textwidth, angle=90]{./img/preguntas/p6.png}
\caption{\label{fig:fig_p6_output_voltage}\footnotesize{Tensión de salida, $V_{o}$, en función de $R_{9}$, con esta variando entre $0 \si[per-mode=symbol]{\ohm}$ y $90 \si[per-mode=symbol]{\kilo\ohm}$.}}
\end{center}
\end{figure}



\clearpage

\subsection{Punto 7}

\textbf{Enunciado}: \textsl{¿Cuál es el rango la corriente de salida de la fuente considerando que $R_{18}$ puede variar desde $0 \si[per-mode=symbol]{\ohm}$ a $18 \si[per-mode=symbol]{\kilo\ohm}$? (Tomar $R_{L} = 0 \si[per-mode=symbol]{\ohm}$).}

\vspace{1.5cm}

El factor de amortiguamiento, $DF$ (damping factor), es la relación entre la impedancia especificada de la carga, $8 \si[per-mode=symbol]{\ohm}$ en este caso, y la impedancia de salida del amplificador, ambas consideradas como resistivas puras, a frecuencias medias, se tiene entonces:


\begin{equation}
\boxed{ DF = \frac{R_{L}}{R_{o}} = \frac{8 \si[per-mode=symbol]{\ohm}}{29.6 \si[per-mode=symbol]{\milli\ohm}} \approx 270.3 }
\end{equation}


\vfill

\clearpage



\subsection{Punto 8}

\textbf{Enunciado}: \textsl{¿Cuál es el valor de la resistencia de carga $R_{L}$ que impone el límite entre el modo fuente de tensión y fuente de corriente para $R_{9} = 90 \si[per-mode=symbol]{\kilo\ohm}$ y $R_{18} = 0 \si[per-mode=symbol]{\ohm}$?.}

\vspace{1.5cm}

\label{punto8}

Para obtener la máxima tensión pico sobre la carga, debemos hacer la suposición que la misma se obtiene al límite donde los transistores del amplificador están al borde salir de modo activo directo, en general no son solo los transistores de la etapa de salida, sino que dependiendo la configuración el límite será impuesto por los transistores de mas de una etapa.\\

En particular para este amplificador podemos ver que para los ciclos positivo de la señal, lo siguiente:

\begin{equation} \label{eq:1}
\hat{V}_{L{max}} + V_{BE_{407}} + V_{BE_{405}} + \hat{V}_{R_{421_{max}}} = V_{CC}
\end{equation}

Condición que lleva a $Q_{407}$ y $Q_{405}$ al límite de la saturación, por supuesto en esta condición la distorsión por alinealidad será  máxima, pero mas allá el amplificador comenzará a recortar.

además tenemos que:

\begin{equation} \label{eq:2}
\hat{V}_{R_{421_{max}}} = \hat{I}_{L_{max}} \cdot R_{421} = \frac{\hat{V}_{L{max}}}{R_{L}}
\end{equation}

Combinando \ref{eq:1} y \ref{eq:2} tenemos:

\begin{equation} \label{eq:3}
\hat{V}_{L{max}} = \frac{V_{CC} - V_{BE_{407}} - V_{BE_{405}} }{1 + \frac{R_{421}}{R_{L}} }
\end{equation}


Pero \ref{eq:3} es una expresión trascendente , ya que $V_{BE_{405}}$ y $V_{BE_{407}}$ dependen de la corriente de colector de cada transistor. Asumiendo para el par una relación de $\beta$ veces en las corrientes, se tiene:


\begin{equation} \label{eq:3}
 \hat{I}_{C_{407_{max}}} = \frac{\hat{I}_{L_{max}}}{1 + \frac{1}{\beta_{407}} } = \frac{\hat{V}_{L_{max}}}{ R_{L} \cdot \left( 1 + \frac{1}{\beta_{407}} \right) }
\end{equation}

Podemos operar iterativamente usando las curvas de $I_{C} = f_{\left( V_{BE} \right)}$ de cada transistor, \textbf{Figura~10},\, \quotemarks{On~Voltages}, de la hoja de datos del transistor \textbf{TIP41}, apéndice~\sectref{datasheet_TIP41}, y \textbf{Figura~4}, \quotemarks{\mbox{Base-Emitter~On~Voltage}}, de la hoja de datos del transistor \textbf{BD135}, apéndice~\sectref{datasheet_BD135}, empezamos asumiendo que $V_{BE} = 0.7$ para ambos transistores, calculamos una tensión de pico máxima y con esta una corriente de colector máxima para el transistor $Q_{407}$ y de esta un nuevo valor para el $V_{BE}$, que nos permite calcular un nuevo valor para la tensión pico máxima, en este caso con una iteración es suficiente, se obtiene:

\begin{equation} \label{eq:3}
\boxed{\hat{V}_{L{max}} = 26.6 \si[per-mode=symbol]{\volt}}
\end{equation}

Para el ciclo negativo se obtiene un valor cercano, haciendo un análisis similar, pero que involucra a los transistores $Q_{406}$ y $Q_{404}$, tomamos este valor para el ciclo positivo, de todos modos es solo una aproximación. 



\vfill

\clearpage


\subsection{Punto 9}

\textbf{Enunciado}: \textsl{¿Qué hace (o para que está) cada componente, o sea, que función cumple en el circuito y justificar el valor de cada resistencia, diodo, transistor, etc?\\
En particular, respecto de la pregunta anterior, explicar que función realiza $D_{1}$ y justificar la elección de su designación como $1N4148$.}


\vspace{1.5cm}

Agrupamos los componentes por sección de la fuente de alimentación.

\subsubsection{Amplificador diferencial con caga activa}

$\bm{Q_{12}}$, $\bm{Q_{13}}$: Transistores \textit{BC548}, forman el par diferencial de la entrada del amplificador, estos transistores en este circuito tienen una ganancia de mas de $2000$, lo que justifica su elección como transistores bipolares \textbf{NPN}, y ya que trabajan con corrientes del orden de $1 \si[per-mode=symbol]{\milli\ampere}$, se necesitan transistores de señal, de propósito general, \textit{BC548} es el mas conocido dentro de las necesidades y categoría.\\

$\bm{Q_{1}}$, $\bm{Q_{2}}$: Transistores \textit{BC558}, forman la fuente espejo con degeneración de emisor que forma la carga del par diferencial, su elección como transistores bipolares \textbf{PNP} se basa en el hecho de ser la carga de un par diferencial con transistores bipolares \textbf{NPN}, su elección como transistores bipolares es para lograr un gran $gm$ que se traduzca en un gran $R_{o}$ para la fuente espejo, y al igual que los transistores del par, trabajan con corrientes del orden de $1 \si[per-mode=symbol]{\milli\ampere}$, por las mismas razones, se necesitan transistores de señal, de propósito general, \textit{BC558} es el mas conocido dentro de las necesidades y categoría, y es el complementario del \textit{BC548}.\\

$\bm{D_{1}}$: Diodo \textit{1N4148}, este diodo actúa durante transitorios, que producen picos de corriente en la base de $Q_{12}$, debido fundamentalmente al switcheo de la llave, durante estos momentos $Q_{12}$ se corre hacia la saturación, lo cual llevaría la tensión de su colector, que es la salida del diferencial, a valores muy bajos, cercanos al del emisor, los cuales tenderían a hacer bajar abruptamente, a través de los transistores de salida, la corriente de salida, lo cual provocaría transitorios mas largos en la respuesta de la fuente, al tener esta que luego subir nuevamente la corriente al actuar la realimentación, el diodo impone un valor de piso de continua de piso, de aproximadamente $0.6 \si[per-mode=symbol]{\volt}$ en el colector de $Q_{12}$ respecto del emisor, limitando cuanto puede esta bajar, y por ende, limitando el efecto anterior. Además de lo antes mencionado, se necesita que dinámicamente no afecte al funcionamiento, siendo idealmente un corto, de ahí la elección de un diodo rápido, \textit{1N4148}, el cual tiene menores efectos reactivos que un diodo normal, se puede decir en pequeña señal que su capacidad equivalente es menor, comportándose para bajas/medias frecuencias como solo una pequeña resistencia de unas pocas decenas de $\si[per-mode=symbol]{\ohm}$.\\

$\bm{R_{1}}$: Resistor de $2.2 \si[per-mode=symbol]{\kilo\ohm}$, resistor que determina la corriente de los transistores del par diferencial, si se considera el punto de reposo con la base de los transistores a $1 \si[per-mode=symbol]{\volt}$, los emisores se encuentran a aproximadamente $0.3 \si[per-mode=symbol]{\volt}$, y sobre este resistor caen aproximadamente $5.3 \si[per-mode=symbol]{\volt}$, con lo que la suma de las corrientes de los transistores del diferencial es de aproximadamente $2.4 \si[per-mode=symbol]{\milli\ampere}$ y la fuente espejo se encarga de repartir esta corriente en forma pareja entre los transistores del par en $1.2 \si[per-mode=symbol]{\milli\ampere}$ aproximadamente. Si la intención del diseño es este valor de corrientes, así queda determinado su valor, las tensiones de alimentación determinan también este valor, por lo cual se incluye esa fuente ideal de $-5 \si[per-mode=symbol]{\volt}$, en un circuito real debería ser implementada con un circuito similar al de la referencia de tensión, el \textit{TL431} puede ser usado en referencias de tensión negativa.\\

$\bm{R_{2}}$, $\bm{R_{3}}$: Resistores de $500 \si[per-mode=symbol]{\ohm}$, estos resistores deben ser iguales para mantener el factor de copia en $1$, y su valor es determinado por el valor de $R_{o}$ que se pretende para la fuente espejo, este valor debe ser lo suficientemente grande como para lograr junto al valor de la resistencia de entrada de la siguiente etapa la ganancia deseada. En este caso el valor que se obtiene es cercano a $1 \si[per-mode=symbol]{\mega\ohm}$, este valor pretendido fija el valor de este resistor.\\

$\bm{R_{6}}$: Resistor de $10 \si[per-mode=symbol]{\ohm}$, este resistor forma parte de una red de compensación, tema de la segunda parte de este trabajo práctico.\\

$\bm{C_{1}}$: Este capacitor forma parte de una de las redes de compensación, tema de la segunda parte del presente trabajo práctico.\\


\subsubsection{Seguidor con carga activa}

$\bm{Q_{3}}$, $\bm{Q_{6}}$: Transistores \textit{BD135}, estos transistores forman la segunda etapa del amplificador, un seguidor por emisor, siendo $\bm{Q_{3}}$ el transistor en seguidor y formando $\bm{Q_{6}}$ su carga activa, una fuente de corriente constante, por lo tanto la etapa no ganará en tensión, pero trabajan con una corriente de alrededor de $17 \si[per-mode=symbol]{\milli\ampere}$, lo que hace que $\bm{Q_{3}}$ disipe una potencia de unos $300 \si[per-mode=symbol]{\milli\watt}$ y $\bm{Q_{6}}$, unos $30 \si[per-mode=symbol]{\milli\watt}$, la intención de esta etapa es presentar una resistencia de entrada lo suficientemente grande al par diferencial para lograr la ganancia buscada, de ahí el usar carga activa, además de independizar su polarización al depender su valor de la fuente estable de $-5 \si[per-mode=symbol]{\volt}$ y del resistor $\bm{R_{8}}$. La elección del transistor está mayormente regida por la potencia disipada en los mismos, ya que es demasiada para un transistor de señal, pero muy poca para un resistor de potencia, eso lleva a elegir un transistor de potencia media, siendo el \textit{BD135}, una elección adecuada, dentro de los transistores de esa categoría, es de los mas conocido y disponible\\

$\bm{R_{8}}$: Resistor de $10 \si[per-mode=symbol]{\ohm}$, este resistor determina junto a la tensión estable de alimentación negativa la corriente de polarización de $\bm{Q_{3}}$ y $\bm{Q_{6}}$.\\

$\bm{R_{34}}$: Resistor de $10 \si[per-mode=symbol]{\ohm}$, este resistor limita la corriente durante transitorios en $\bm{Q_{3}}$ y $\bm{Q_{6}}$, su valor se determina de manera empírica por simulación y/o medición, no puede ser muy grande porque disiparía mucha potencia y mas importante comenzaría a disminuir la ganancia y la resistencia de entrada de $\bm{Q_{3}}$.\\


\subsubsection{Referencia de tensión}

$\bm{U_{2}}$, $\bm{Q_{14}}$, $\bm{R_{30}}$, $\bm{R_{31}}$, $\bm{R_{32}}$, $\bm{R_{33}}$, $\bm{C_{12}}$: Referencia de tensión basada en el \textit{TL431}, esta se analiza en detalle en la sección~\sectref{section:voltage_reference}. Solo queda aclarar el valor específico de $\bm{R_{31}}$ y $\bm{R_{32}}$, sus valores están ajustados a valores comerciales para obtener $10 \si[per-mode=symbol]{\volt}$ de referencia.\\


$\bm{R_{4}}$, $\bm{R_{55}}$, $\bm{R_{5}}$: Estos tres resistores forman un divisor resistivo que genera a partir de la referencia de $10 \si[per-mode=symbol]{\volt}$  una de $1 \si[per-mode=symbol]{\volt}$, los valores se eligen de manera de lograr esta tensión y cargar lo menos posible a la referencia, teniendo en cuenta que la carga de la base de $\bm{Q_{13}}$ debe ser despreciable, que es del orden de $4 \si[per-mode=symbol]{\micro\ampere}$, con estos valores tenemos una corriente de $0.9 \si[per-mode=symbol]{\milli\ampere}$, logrando una tensión de prácticamente $1 \si[per-mode=symbol]{\volt}$ sobre la base de $\bm{Q_{13}}$. Los valores son tales de lograr lo anterior, pero se usan tres resistores por la precisión necesaria, lo que por supuesto implica que deben ser de baja tolerancia, en particular estos valores están en la serie de 1 \%.\\



\subsubsection{Par compuesto (Sziklai)}

$\bm{Q_{4}}$, $\bm{Q_{5}}$, $\bm{R_{11}}$: Par compuesto (Sziklai), explicado en detalle en la sección~\sectref{section:sziklai}.\\


\subsubsection{Limitación de corriente simple}

$\bm{Q_{15}}$, $\bm{R_{S}}$: Este circuito es una simple limitación de corriente que limita quitando la corriente a la base del primer transistor del par compuesto, simplemente al llegar la corriente a un valor tal que la caída sobre $\bm{R_{S}}$ sea la necesaria para que $\bm{Q_{15}}$ comience a conducir, se tendrá limitación de corriente, este valor corresponde a aproximadamente $3 \si[per-mode=symbol]{\ampere}$. Dado que la fuente de alimentación cuenta con un lazo de corriente, este circuito solo está para limitar la corriente durante transitorios, protegiendo de estos picos de corriente a los transistores y a la fuente de entrada.\\

\subsubsection{Realimentación de corriente}

$\bm{U_{3_{A}}}$, $\bm{U_{3_{B}}}$, $\bm{R_{12}}$, $\bm{R_{13}}$, $\bm{R_{14}}$, $\bm{R_{15}}$, $\bm{R_{16}}$, $\bm{R_{17}}$, $\bm{R_{18}}$, $\bm{R_{19}}$, $\bm{R_{S}}$: Este circuito forma un amplificador diferencial y un amplificador no inversor en cascada, los cuales se explican en detalle en  la sección~\sectref{section:modelo_operacional}. Además de esto al tener en cuenta $\bm{R_{S}}$, tenemos que el amplificador diferencial convierte corriente a tensión, y dado que el valor de $\bm{R_{S}}$, $0.2 \si[per-mode=symbol]{\ohm}$ es despreciable frente a la resistencia de entrada del diferencial, aproximadamente $27 \si[per-mode=symbol]{\kilo\ohm}$, el factor de conversión de corriente a tensión es simplemente el valor de resistencia de $\bm{R_{S}}$.\\

$\bm{C_{15}}$, $\bm{C_{16}}$: Estos capacitores forman parte de redes de compensación, tema de la segunda parte del presente trabajo práctico.\\

\subsubsection{Realimentación de Tensión}

$\bm{R_{9}}$, $\bm{R_{10}}$: Estos dos resistores forman un divisor resistivo que es básicamente la red de realimentación, si no se tiene en cuenta la llave que es trasparente. Sus valores, o rango de valores mas bien, son tales de lograr una realimentación que va de $0.1$ a $1$, logrando una ganancia de tensión a lazo cerrado que va de $1$ a $10$.\\

$\bm{R_{60}}$, $\bm{C_{9}}$, $\bm{C_{2}}$: Estos capacitores forman parte de redes de compensación, tema de la segunda parte del presente trabajo práctico.\\


\subsubsection{Llave analógica}

$\bm{Q_{11}}$, $\bm{Q_{7}}$, $\bm{Q_{9}}$, $\bm{Q_{10}}$, $\bm{R_{20}}$, $\bm{R_{21}}$, $\bm{R_{22}}$, $\bm{R_{23}}$, $\bm{R_{7}}$: Llave electrónica transparente, explicada en detalle en la sección~\sectref{section:switch}. Quedaría analizar el uso de estos específicos transistores y resistores. El uso de los transistores \textit{BC548} y \textit{BC558}, básicamente se debe a que son necesarios transistores de señal y complementarios para lograr la mejor compensación de sus caídas $V_{BE}$, lo cual como antes nos lleva a estos transistores como elección. Los valores de los resistores son tales de lograr unas resistencias de entrada de los seguidores lo suficientemente alta como para despreciarse el efecto de carga sobre el divisor resistivo y al mismo tiempo no comprometer su respuesta en frecuencia, ni su resistencia de salida. En particular el valor de $\bm{R_{7}}$, se establece para compensar el offset del diferencial por la caída en la resistencia vista por las bases de $\bm{Q_{12}}$ y $\bm{Q_{13}}$, dado que $\bm{Q_{12}}$ ve aproximadamente $1 \si[per-mode=symbol]{\kilo\ohm}$, esto fija el valor para  $\bm{R_{7}}$.\\

\subsubsection{Alimentación}

$\bm{C_{11}}$, $\bm{C_{13}}$: Estos capacitores, de $10 \si[per-mode=symbol]{\micro\farad}$, son capacitores de filtro de la alimentación, ayudando a filtrar un posible rizado presente en la fuente de alimentación de entrada y en la salida de la referencia respectivamente.\\



\clearpage


\subsection{Punto 10}

\textbf{Enunciado}: \textsl{¿Qué tecnología, tolerancia, capacidad de disipación de potencia, estabilidad con la temperatura, tensión y corriente de operación máxima y pulsante, características mecánicas, apartamiento de su valor nominal por envejecimiento, etc, debe tener cada componente considerando una implementación física de éste circuito?.}


\vspace{1.5cm}


Agrupamos los componentes por sección de la fuente de alimentación.

\subsubsection{Amplificador diferencial con caga activa}

$\bm{Q_{12}}$, $\bm{Q_{13}}$: \textit{BC548}. Transistor de silicio \textbf{NPN} de señal. Encapsulado TO-92 si se usa tecnología \quotemarks{through hole}.\\

$\bm{Q_{1}}$, $\bm{Q_{2}}$: \textit{BC558}. Transistor de silicio \textbf{PNP} de señal. Encapsulado TO-92 si se usa tecnología \quotemarks{through hole}.\\

$\bm{D_{1}}$: \textit{1N4148}. Diodo de silicio transición rápida. Encapsulado DO-35 si se usa tecnología \quotemarks{through hole}.\\

$\bm{R_{1}}$, $\bm{R_{2}}$, $\bm{R_{3}}$, $\bm{R_{7}}$: Resistor de película de metal u óxido de metal, $\frac{1}{4} \si[per-mode=symbol]{\watt}$, con una tolerancia de $1 \%$.\\

$\bm{R_{6}}$: Resistor de película de carbón, $\frac{1}{4} \si[per-mode=symbol]{\watt}$, con una tolerancia de $5 \%$.\\

$\bm{C_{1}}$: Capacitor cerámico o polyester.\\


\subsubsection{Seguidor con carga activa}

$\bm{Q_{3}}$, $\bm{Q_{6}}$: \textit{BD135}. Transistor de silicio \textbf{NPN} de potencia media. Encapsulado TO-225 si se usa tecnología \quotemarks{through hole}.\\
$\bm{R_{8}}$: Resistor de película de metal u óxido de metal, $\frac{1}{4} \si[per-mode=symbol]{\watt}$, con una tolerancia de $1 \%$.\\

$\bm{R_{34}}$: Resistor de película de carbón, $\frac{1}{4} \si[per-mode=symbol]{\watt}$, con una tolerancia de $5 \%$.\\



\subsubsection{Referencia de tensión}

$\bm{U_{2}}$: Referencia de tensión programable con bajo coeficiente térmico. encapsulado SOT-23 si se usa tecnología \quotemarks{through hole}.\\



$\bm{Q_{14}}$: \textit{BD135}. Transistor de silicio \textbf{NPN} de potencia media. Encapsulado TO-225 si se usa tecnología \quotemarks{through hole}.\\


$\bm{R_{31}}$, $\bm{R_{32}}$, $\bm{R_{4}}$, $\bm{R_{55}}$, $\bm{R_{5}}$: Resistor de película de metal u óxido de metal, $\frac{1}{4} \si[per-mode=symbol]{\watt}$, con una tolerancia de $1 \%$.\\


$\bm{R_{30}}$, $\bm{R_{33}}$: Resistor de película de carbón, $\frac{1}{4} \si[per-mode=symbol]{\watt}$, con una tolerancia de $5 \%$.\\


$\bm{C_{12}}$: Capacitor cerámico o polyester.\\


 \subsubsection{Par compuesto (Sziklai)}

$\bm{Q_{4}}$: \textit{MJE15032}. Transistor de silicio \textbf{NPN} de potencia media, alta ganancia y ancho de banda. Encapsulado TO-220 si se usa tecnología \quotemarks{through hole}.\\

$\bm{Q_{5}}$: \textit{MJE2955}. Transistor de silicio \textbf{PNP} de potencia, alta ganancia y ancho de banda. Encapsulado TO-220 o TO-3 si se usa tecnología \quotemarks{through hole}.\\

$\bm{R_{11}}$: Resistor de película de carbón, $\frac{1}{4} \si[per-mode=symbol]{\watt}$, con una tolerancia de $5 \%$.\\


\subsubsection{Limitación de corriente simple}

$\bm{Q_{15}}$: \textit{BD135}. Transistor de silicio \textbf{NPN} de potencia media. Encapsulado TO-225 si se usa tecnología \quotemarks{through hole}.\\

$\bm{R_{S}}$: Resistor de alambre, encapsulado cerámico, de $2 \si[per-mode=symbol]{\watt}$, con una tolerancia de $1 \%$.\\


\subsubsection{Realimentación de corriente}

$\bm{U_{3_{A}}}$, $\bm{U_{3_{B}}}$: Amplificador operacional de bajo ruido con entradas \textbf{JFET}.\\


$\bm{R_{12}}$, $\bm{R_{13}}$, $\bm{R_{14}}$, $\bm{R_{15}}$, $\bm{R_{16}}$, $\bm{R_{17}}$, $\bm{R_{18}}$, $\bm{R_{19}}$: Resistor de película de metal u óxido de metal, $\frac{1}{4} \si[per-mode=symbol]{\watt}$, con una tolerancia de $1 \%$.\\


$\bm{C_{15}}$, $\bm{C_{16}}$: Capacitor cerámico o polyester.\\ 

\subsubsection{Realimentación de Tensión}

$\bm{R_{9}}$, $\bm{R_{10}}$: Resistor de película de metal u óxido de metal, $\frac{1}{4} \si[per-mode=symbol]{\watt}$, con una tolerancia de $1 \%$.\\

$\bm{R_{60}}$: Resistor de película de carbón, $\frac{1}{4} \si[per-mode=symbol]{\watt}$, con una tolerancia de $5 \%$.\\

$\bm{C_{9}}$, $\bm{C_{2}}$: Capacitor cerámico o polyester.\\ 

\subsubsection{Llave analógica}

$\bm{Q_{11}}$, $\bm{Q_{10}}$: \textit{BC558}. Transistor de silicio \textbf{PNP} de señal. Encapsulado TO-92 si se usa tecnología \quotemarks{through hole}.\\

$\bm{Q_{7}}$, $\bm{Q_{9}}$: \textit{BC548}. Transistor de silicio \textbf{NPN} de señal. Encapsulado TO-92 si se usa tecnología \quotemarks{through hole}.\\


$\bm{R_{20}}$, $\bm{R_{21}}$, $\bm{R_{22}}$, $\bm{R_{23}}$: Resistor de película de metal u óxido de metal, $\frac{1}{4} \si[per-mode=symbol]{\watt}$, con una tolerancia de $1 \%$.\\



\subsubsection{Alimentación}

$\bm{C_{11}}$: Capacitor electrolítico de aluminio de $25 \si[per-mode=symbol]{\volt}$, con una tolerancia de $20 \%$.\\

 $\bm{C_{13}}$: Capacitor electrolítico de aluminio de $50 \si[per-mode=symbol]{\volt}$, con una tolerancia de $20 \%$.\\








\vfill

\clearpage


\subsection{Punto 11}

\textbf{Enunciado}: \textsl{Calcular la ganancia de lazo \quotemarks{af} para el lazo de tensión y para el lazo de corriente, comparando en ambos casos con respecto a 1, o sea, ¿Resulta \quotemarks{af} mucho mayor que $1$? Considerar esto para frecuencias del orden de entre $0 \si[per-mode=symbol]{\hertz}$ y $100 \si[per-mode=symbol]{\hertz}$.}

\vspace{1.5cm}

\clearpage

\subsubsection{Punto de reposo hallado por simulación}

En la figura~\figref{fig:fig_simulated_qpoint} se muestra lo hallado por simulación para el circuito. Vemos una gran similitud con los valores calculados anteriormente, figura~\figref{fig:fig_calculated_qpoint}, aunque hubo que cambiar el valor de $PS_{401}$ con respecto al calculado para acercarnos lo más posible a $0 \si[per-mode=symbol]{\volt}$ en la salida, y también se tuvo que cambiar el valor de $PS_{402_{A}}$ y $PS_{402_{B}}$ para asegurar $10 \si[per-mode=symbol]{\milli\ampere}$ en los colectores de $Q_{407}$ y $Q_{408}$.


\begin{figure}[H] %htb
\begin{center}
\includegraphics[width=0.9 \textwidth, angle=90]{./img/circuitos_usados/P1_P11a_qpoint.png}
\caption{\label{fig:fig_simulated_qpoint}\footnotesize{Punto de reposo del circuito hallado por simulación.}}
\end{center}
\end{figure}


\clearpage

\subsubsection{Impedancia de entrada hallada por simulación}

En la figura~\figref{fig:fig_simulated_zi} se muestra lo obtenido al simular para obtener la impedancia de entrada, el valor a frecuencias medias $86.02 \si[per-mode=symbol]{\kilo\ohm}$ se acerca bastante al valor calculado en la sección~\sectref{calculated_zi}, se había calculado $85.3 \si[per-mode=symbol]{\kilo\ohm}$.


\begin{figure}[H] %htb
\begin{center}
\includegraphics[width=0.9 \textwidth, angle=90]{./img/puntos/P11b_Ri.png}
\caption{\label{fig:fig_simulated_zi}\footnotesize{Impedancia de entrada hallada por simulación.}}
\end{center}
\end{figure}



\clearpage

\subsubsection{Impedancia de salida hallada por simulación}

En la figura~\figref{fig:fig_simulated_zo} se muestra lo obtenido al simular para obtener la impedancia de salida, el valor a frecuencias medias $17.58 \si[per-mode=symbol]{\milli\ohm}$ se aparta un poco del valor calculado en la sección~\sectref{calculated_zo}, se había calculado $29.6 \si[per-mode=symbol]{\milli\ohm}$, pero por lo dicho en dicha sección acerca del cálculo de la impedancia de salida a lazo abierto y por la fuerte dependencia del valor con la ganancia de lazo, que se calcula en forma aproximada, era esperable la diferencia.


\begin{figure}[H] %htb
\begin{center}
\includegraphics[width=0.9 \textwidth, angle=90]{./img/puntos/P11c_Ro.png}
\caption{\label{fig:fig_simulated_zo}\footnotesize{Impedancia de salida hallada por simulación.}}
\end{center}
\end{figure}



\clearpage

\subsubsection{Respuesta en frecuencia para $1 \si[per-mode=symbol]{\watt}$ sobre la carga}

En la figura~\figref{fig:fig_RF_1W} se muestra lo obtenido al simular para obtener la respuesta en frecuencia a $1 \si[per-mode=symbol]{\watt}$ de potencia sobre la carga, en la misma se puede ver el valor del ancho de banda encontrado, $129.55 \si[per-mode=symbol]{\kilo\hertz}$.

\begin{figure}[H] %htb
\begin{center}
\includegraphics[width=0.9 \textwidth, angle=90]{./img/puntos/P11d_RF_1W.png}
\caption{\label{fig:fig_RF_1W}\footnotesize{Respuesta en frecuencia para $1 \si[per-mode=symbol]{\watt}$ sobre la carga.}}
\end{center}
\end{figure}


\clearpage

\subsubsection{Ancho de banda de potencia, a máxima potencia sobre la carga}

En la figura~\figref{fig:fig_RF_MAX_POWER} se muestra lo obtenido al simular para obtener la respuesta en frecuencia a $1 \si[per-mode=symbol]{\watt}$ de potencia sobre la carga, en la misma se puede ver el valor del ancho de banda encontrado, $129.55 \si[per-mode=symbol]{\kilo\hertz}$, idéntico que para el caso de $1 \si[per-mode=symbol]{\watt}$ .

\begin{figure}[H] %htb
\begin{center}
\includegraphics[width=0.9 \textwidth, angle=90]{./img/puntos/P11e_Power_BW.png}
\caption{\label{fig:fig_RF_MAX_POWER}\footnotesize{Respuesta en frecuencia para máxima potencia sobre la carga.}}
\end{center}
\end{figure}


\clearpage

\subsubsection{Respuesta al escalón en pequeña señal}

En la figura~\figref{fig:fig_step_small_signal} se muestra lo obtenido al simular para obtener la respuesta al escalón en pequeña señal, se limitó la salida a un valor de $1 \si[per-mode=symbol]{\volt}$ de pico. La respuesta tiene la forma esperada para un amplificador con acoples capacitivos.

\begin{figure}[H] %htb
\begin{center}
\includegraphics[width=0.9 \textwidth, angle=90]{./img/puntos/P11f_I_step_small_signal.png}
\caption{\label{fig:fig_step_small_signal}\footnotesize{Respuesta al escalón en pequeña señal.}}
\end{center}
\end{figure}


En la figura~\figref{fig:fig_step_small_signal_zoom} se muestra la ampliación del flanco ascendente de la salida, donde se puede apreciar el tiempo de crecimiento, usando la directiva de \textbf{SPICE}, \textit{.measure}, se calculó directamente de la simulación el tiempo de crecimiento entre $10\%$ y $90\%$ y se computó en base a este el ancho de banda del circuito. Se utilizó la expresión que relaciona ancho de banda con el tiempo de crecimiento en un circuito con un solo polo:

\begin{equation}
BW = \frac{0.35}{T_{rise}}
\end{equation}

Se obtuvo:

\begin{equation}
T_{rise} = 2.78 \si[per-mode=symbol]{\micro\second}
\end{equation}


\begin{equation}
\boxed{BW = 125.863 \si[per-mode=symbol]{\kilo\hertz}}
\end{equation}



\begin{figure}[H] %htb
\begin{center}
\includegraphics[width=0.9 \textwidth, angle=90]{./img/puntos/P11f_I_step_small_signal_zoom.png}
\caption{\label{fig:fig_step_small_signal_zoom}\footnotesize{Respuesta al escalón en pequeña señal, ampliación del flanco.}}
\end{center}
\end{figure}

\clearpage

\subsubsection{Respuesta al escalón en gran señal}

En la figura~\figref{fig:fig_step_big_signal} se muestra lo obtenido al simular para obtener la respuesta al escalón en gran señal, se llevó la salida a un valor cercano al máximo sin distorsión.

\begin{figure}[H] %htb
\begin{center}
\includegraphics[width=0.9 \textwidth, angle=90]{./img/puntos/P11f_I_step_big_signal.png}
\caption{\label{fig:fig_step_big_signal}\footnotesize{Respuesta al escalón en gran señal.}}
\end{center}
\end{figure}


En la figura~\figref{fig:fig_step_big_signal_zoom} se muestra la ampliación del flanco ascendente de la salida, donde se puede apreciar el tiempo de crecimiento, usando la directiva de \textbf{SPICE}, \textit{.measure}, se calculó directamente de la simulación el \textbf{\quotemarks{slew~rate}}, como la pendiente de subida en el flanco ascendente, se obtuvo:


\begin{equation}
\boxed{SR = 4.39 \si[per-mode=symbol]{\volt\per\micro\second}}
\end{equation}



\begin{figure}[H] %htb
\begin{center}
\includegraphics[width=0.9 \textwidth, angle=90]{./img/puntos/P11f_I_step_big_signal_zoom.png}
\caption{\label{fig:fig_step_big_signal_zoom}\footnotesize{Respuesta al escalón en gran señal, ampliación del flanco.}}
\end{center}
\end{figure}

\clearpage


\subsubsection{Margen de fase del amplificador}

En la figura~\figref{fig:fig_phase_margin} se muestra lo obtenido al simular para obtener el margen de fase del circuito, para esta simulación se abrió el lazo de realimentación para la señal y se tomó la respuesta de la cascada del amplificador con la realimentación, en la figura~\figref{fig:fig_loop_gain_circuit}, se puede ver el circuito utilizado para esta simulación. El valor obtenido para el margen de fase es:

\begin{equation}
\boxed{PM = 113.72 \si[per-mode=symbol]{\degree}}
\end{equation}

\begin{figure}[H] %htb
\begin{center}
\includegraphics[width=0.9 \textwidth, angle=90]{./img/puntos/P11g_phase_margin.png}
\caption{\label{fig:fig_phase_margin}\footnotesize{Margen de fase.}}
\end{center}
\end{figure}

\begin{figure}[H] %htb
\begin{center}
\includegraphics[width=0.9 \textwidth, angle=90]{./img/circuitos_usados/P11g_phase_margin.png}
\caption{\label{fig:fig_loop_gain_circuit}\footnotesize{Circuito usado para simular la ganancia de lazo.}}
\end{center}
\end{figure}

\clearpage

\subsubsection{Distorsión armónica del amplificador}

En el cuadro~\tableref{table:table_THD} se resumen los resultados obtenidos al realizar la simulación para determinar la distorsión armónica total (\textbf{THD}) para 8 combinaciones de frecuencia y potencia de salida sobre la carga. El cálculo se realizo directamente con el comando \textbf{SPICE} \textit{.fourier}, teniendo en cuenta nueve armónicas de la señal y usando todos los datos de aproximadamente $1\si[per-mode=symbol]{\second}$ de simulación.



%% \noindent
%% \begin{center}
 
%%\begin{spacing}{1}  
\begin{table}[H]  %%\centering
    
    \setlength\arrayrulewidth{1.5pt}
    \arrayrulecolor{white}
    \def\clinecolor{\hhline{|>{\arrayrulecolor{white}}-%
    >{\arrayrulecolor{white}}|-|-|-|-|}}
\resizebox{0.8 \textwidth}{!}{% 
       
\begin{tabularx}{1 \textwidth}%
    {|
    >{\columncolor{white} \centering\arraybackslash}m{0.32\linewidth}
     |
    >{\columncolor{white} \centering\arraybackslash}m{0.17\linewidth}
     |
    >{\columncolor{white} \centering\arraybackslash}m{0.17\linewidth}
     |
    >{\columncolor{white} \centering\arraybackslash}m{0.17\linewidth}
     |
    >{\columncolor{white} \centering\arraybackslash}m{0.17\linewidth}
     |
    }
    \rowcolor{HeadersColor} \cellcolor{white} \thead{}  & \thead{$0.1 \si[per-mode=symbol]{\watt}$} & \thead{$1 \si[per-mode=symbol]{\watt}$} & \thead{$10 \si[per-mode=symbol]{\watt}$} & \thead{$90 \%$ de max.} \\    
    \hhline{|-|-|-|-|}
    \rowcolor{gray!20} \cellcolor{HeadersColor} \color{white} $1 \si[per-mode=symbol]{\kilo\hertz}$ & $0.055\%$ & $0.023\%$ & $0.014\%$ & $0.055\%$  \\
    \hhline{|-|-|-|-|}
    \rowcolor{gray!20} \cellcolor{HeadersColor} \color{white} $10 \si[per-mode=symbol]{\kilo\hertz}$ & $0.144\%$ & $0.077\%$ & $0.057\%$ & $0.107\%$   \\
    \hhline{|-|-|-|-|}       
    \end{tabularx}}
	\caption{\footnotesize{Distorsión armónica total (\textbf{THD}).}}
	\label{table:table_THD}
\end{table}
%%\end{spacing}

%% \end{center}

Vemos que la distorsión es menor para $1 \si[per-mode=symbol]{\kilo\hertz}$  de frecuencia de entrada y también que presenta un mínimo alrededor de las potencias medias, es decir, disminuye de bajas a medias potencias y sube de medias a altas potencias.


\clearpage

\subsubsection{Distorsión por intermodulación del amplificador}

En el cuadro~\tableref{table:table_IMD} se resumen los resultados obtenidos al realizar la simulación para determinar la distorsión por intermodulación (\textbf{IMD}) para 4 potencias de salida sobre la carga. El cálculo se realizo con el comando \textbf{SPICE} \textit{.fourier}, se tomaron las armónicas de $100 \si[per-mode=symbol]{\hertz}$ hasta la armónica $55$, de modo de tomar $5$ armónicas por arriba y $5$ armónicas por debajo del tono puro de $5 \si[per-mode=symbol]{\kilo\hertz}$ , y usando todos los datos de aproximadamente $1\si[per-mode=symbol]{\second}$ de simulación.\\
Se observa que la \textbf{IMD} parece crecer para valores bajos y altos de la potencia de salida, teniendo un mínimo a potencias medias.



%% \noindent
%% \begin{center}
 
%%\begin{spacing}{1}  
\begin{table}[H]  %%\centering
    
    \setlength\arrayrulewidth{1.5pt}
    \arrayrulecolor{white}
    \def\clinecolor{\hhline{|>{\arrayrulecolor{white}}-%
    >{\arrayrulecolor{white}}|-|-|-|-|}}
\resizebox{0.8 \textwidth}{!}{% 
       
\begin{tabularx}{1 \textwidth}%
    {|
    >{\columncolor{white} \centering\arraybackslash}m{0.32\linewidth}
     |
    >{\columncolor{white} \centering\arraybackslash}m{0.17\linewidth}
     |
    >{\columncolor{white} \centering\arraybackslash}m{0.17\linewidth}
     |
    >{\columncolor{white} \centering\arraybackslash}m{0.17\linewidth}
     |
    >{\columncolor{white} \centering\arraybackslash}m{0.17\linewidth}
     |
    }
    \rowcolor{HeadersColor} \cellcolor{white} \thead{}  & \thead{$0.1 \si[per-mode=symbol]{\watt}$} & \thead{$1 \si[per-mode=symbol]{\watt}$} & \thead{$10 \si[per-mode=symbol]{\watt}$} & \thead{$90 \%$ de max.} \\    
    \hhline{|-|-|-|-|}
    \rowcolor{gray!20} \cellcolor{HeadersColor} \color{white} \textbf{IMD} & $0.108 \%$ & $0.043 \%$ & $0.048 \%$ & $0.51 \%$ \\
    \hhline{|-|-|-|-|}     
    \end{tabularx}}
	\caption{\footnotesize{Distorsión armónica total (\textbf{IMD}).}}
	\label{table:table_IMD}
\end{table}
%%\end{spacing}

%% \end{center}



\clearpage

\subsubsection{Rechazo de Ruido de la Fuente de Alimentación (\textbf{\quotemarks{PSNR}}).}

En el cuadro~\tableref{table:table_PSNR} se resumen los resultados obtenidos al realizar la simulación para determinar el rechazo de ruido de la fuente de alimentación (\textbf{PSNR}) para 4 frecuencias de la señal de ruido presente en la fuente de alimentación y para una tensión de pico de ruido de $1 \si[per-mode=symbol]{\milli\volt}$.



%% \noindent
%% \begin{center}
 
%%\begin{spacing}{1}  
\begin{table}[H]  %%\centering
    
    \setlength\arrayrulewidth{1.5pt}
    \arrayrulecolor{white}
    \def\clinecolor{\hhline{|>{\arrayrulecolor{white}}-%
    >{\arrayrulecolor{white}}|-|-|-|-|-|-|}}
\resizebox{0.8 \textwidth}{!}{% 
       
\begin{tabularx}{1 \textwidth}%
    {|
    >{\columncolor{white} \centering\arraybackslash}m{0.25\linewidth}
     |
    >{\columncolor{white} \centering\arraybackslash}m{0.125\linewidth}
     |
    >{\columncolor{white} \centering\arraybackslash}m{0.125\linewidth}
     |
    >{\columncolor{white} \centering\arraybackslash}m{0.125\linewidth}
     |
    >{\columncolor{white} \centering\arraybackslash}m{0.125\linewidth}
     |
    >{\columncolor{white} \centering\arraybackslash}m{0.125\linewidth}
     |
    >{\columncolor{white} \centering\arraybackslash}m{0.125\linewidth}
     |
    }
    \rowcolor{HeadersColor} \cellcolor{white} \thead{}  & \thead{$50 \si[per-mode=symbol]{\hertz}$} & \thead{$100 \si[per-mode=symbol]{\hertz}$} & \thead{$1 \si[per-mode=symbol]{\kilo\hertz}$} & \thead{ $10 \si[per-mode=symbol]{\kilo\hertz}$} & \thead{$50 \si[per-mode=symbol]{\kilo\hertz}$} & \thead{$100 \si[per-mode=symbol]{\kilo\hertz}$}\\    
    \hhline{|-|-|-|-|-|-|}
    \rowcolor{gray!20} \cellcolor{HeadersColor} \color{white} \textbf{PSNR} & $ 53.2 \si[per-mode=symbol]{\decibel} $ & $ 59.1 \si[per-mode=symbol]{\decibel} $ & $ 78.99 \si[per-mode=symbol]{\decibel} $ & $ 93.07 \si[per-mode=symbol]{\decibel} $ & $ 89.18 \si[per-mode=symbol]{\decibel} $ & $ 84.82 \si[per-mode=symbol]{\decibel} $ \\
    \hhline{|-|-|-|-|-|-|}     
    \end{tabularx}}
	\caption{\footnotesize{Rechazo de Ruido de la Fuente de Alimentación (\textbf{\quotemarks{PSNR}}).}}
	\label{table:table_PSNR}
\end{table}
%%\end{spacing}

%% \end{center}

El rechazo al ruido de la fuente parece ser mayor cerca del centro de la banda del amplificador.


\subsection{Punto 12}

\textbf{Enunciado}: \textsl{Calcular la impedancia de salida, o más propiamente la impedancia en el nodo de salida, para una carga de $100 \si[per-mode=symbol]{\ohm}$ y una frecuencia en el entorno a $50 \si[per-mode=symbol]{\hertz}$. Utilizar para el cálculo los mismo modelos utilizados en la pregunta anterior.}


\vspace{1.5cm}


Se explica con detalle en las secciones~\sectref{section:voltage_loop}~y~~\sectref{section:current_loop}.


\subsection{Punto 13}

\textbf{Enunciado}: \textsl{Hallar por simulación la impedancia del nodo de salida en función de la frecuencia para frecuencias desde $0,1 \si[per-mode=symbol]{\hertz}$ hasta $100 \si[per-mode=symbol]{\kilo\hertz}$ y con $R_{L} = 100 \si[per-mode=symbol]{\ohm}$. Considerar $R_{9} = 10 \si[per-mode=symbol]{\kilo\ohm}$.}


\vspace{1.5cm}


En la figura~\figref{fig:fig_p13_output_impedance} se muestra el gráfico de la impedancia en el nodo de salida en modo de regulación de tensión, el gráfico se obtuvo simulando en \textbf{SPICE} con $R_{L} = 100 \si[per-mode=symbol]{\ohm}$, con una fuente de corriente de señal conectada en paralelo con la carga, $I_{p}$, realizando un barrido en alterna de $0.1 \si[per-mode=symbol]{\hertz}$ a $100 \si[per-mode=symbol]{\kilo\hertz}$, con el comando \textbf{SPICE} \textit{.ac}, y luego obteniendo el cociente $\frac{V\left(I_{p}\right)}{I_{p}}$, el resultado se exportó y se graficó en \textbf{MATLAB}, en escala semilogarítmica, su módulo y su fase, se destacó el resultado a bajas frecuencias que representa la resistencia de salida a frecuencia bajas/medias. El bajo valor obtenido para esta resistencia ($473 \si[per-mode=symbol]{\micro\ohm}$) implica que se trata de una buena fuente de tensión, que en el caso ideal tiene resistencia de salida de $0 \si[per-mode=symbol]{\ohm}$, esto se debe a la gran ganancia de lazo en modo de regulación de tensión. Otra cosa que se puede observar es que al aumentar la frecuencia la impedancia aumenta, al caer la ganancia de lazo, y se torna inductiva, al menos hasta que la fase supera los $90 \si[per-mode=symbol]{\degree}$, esto parece indicar un efecto de resistencia negativa, la fuente entregaría energía de alterna (esto necesita mas análisis).




\vfill

\clearpage

\begin{figure}[H] %htb
\begin{center}
\includegraphics[width=1.2 \textwidth, angle=90]{./img/preguntas/p13.png}
\caption{\label{fig:fig_p13_output_impedance}\footnotesize{Impedancia de salida, $Z_{o}$, en función de la frecuencia, con esta variando entre $0.1 \si[per-mode=symbol]{\hertz}$ y $100 \si[per-mode=symbol]{\kilo\hertz}$.}}
\end{center}
\end{figure}



\clearpage


\subsection{Punto 14}

\textbf{Enunciado}: \textsl{Hallar por simulación la impedancia de la malla de salida en función de la frecuencia para frecuencias desde $0,1 \si[per-mode=symbol]{\hertz}$ hasta $100 \si[per-mode=symbol]{\kilo\hertz}$ y con  $R_{L} = 0 \si[per-mode=symbol]{\ohm}$. Considerar  $R_{18} = 0 \si[per-mode=symbol]{\ohm}$.}


\vspace{1.5cm}


En la figura~\figref{fig:fig_p14_output_impedance} se muestra el gráfico de la impedancia en el nodo de salida en modo de regulación de corriente, el gráfico se obtuvo simulando en \textbf{SPICE} con la salida cortocircuitada a través de una fuente de tensión de señal, $V_{p}$, realizando un barrido en alterna de $0.1 \si[per-mode=symbol]{\hertz}$ a $100 \si[per-mode=symbol]{\kilo\hertz}$, con el comando \textbf{SPICE} \textit{.ac}, y luego obteniendo el cociente $\frac{V_{p}}{I\left({V_{p}}\right)}$, el resultado se exportó y se graficó en \textbf{MATLAB}, en escala semilogarítmica, su módulo y su fase, se destacó el resultado a bajas frecuencias que representa la resistencia de salida a frecuencia bajas/medias. El bajo valor obtenido para esta resistencia ($958 \si[per-mode=symbol]{\ohm}$) implica que no se trata de una buena fuente de corriente, que en el caso ideal tiene resistencia de salida $\infty$, esto se debe a la menor ganancia de lazo en modo regulación de corriente respecto al modo regulación tensión. Otra cosa que se puede observar es que al aumentar la frecuencia la impedancia disminuye, al caer la ganancia de lazo, y se torna capacitiva.




\vfill

\clearpage

\begin{figure}[H] %htb
\begin{center}
\includegraphics[width=1.2 \textwidth, angle=90]{./img/preguntas/p14.png}
\caption{\label{fig:fig_p14_output_impedance}\footnotesize{Impedancia de salida, $Z_{o}$, en función de la frecuencia, con esta variando entre $0.1 \si[per-mode=symbol]{\hertz}$ y $100 \si[per-mode=symbol]{\kilo\hertz}$.}}
\end{center}
\end{figure}



\clearpage


\subsection{Punto 15}

\textbf{Enunciado}: \textsl{Hallar por simulación la tensión del nodo de salida en función de la corriente de salida para  $R_{L}$ variando entre  $100 \si[per-mode=symbol]{\ohm}$ y $0 \si[per-mode=symbol]{\ohm}$. Considerar $R_{9} = 10 \si[per-mode=symbol]{\kilo\ohm}$ y $R_{18} = 0 \si[per-mode=symbol]{\ohm}$.}


\vspace{1.5cm}


En la figura~\figref{fig:fig_p15_voltage_vs_current} se muestra la variación de la tensión de salida en función de la corriente de salida, se distinguen claramente y están marcadas, las regiones de regulación de tensión (la tensión nominal esperada es de $2 \si[per-mode=symbol]{\volt}$) y corriente (la corriente esperada es de $2 \si[per-mode=symbol]{\ampere}$).




\vfill

\clearpage

\begin{figure}[H] %htb
\begin{center}
\includegraphics[width=1.2 \textwidth, angle=90]{./img/preguntas/p15.png}
\caption{\label{fig:fig_p15_voltage_vs_current}\footnotesize{Tensión de salida en función de la corriente de salida para $R_{L}$ variando entre $100 \si[per-mode=symbol]{\ohm} y 0 \si[per-mode=symbol]{\ohm}$.}}
\end{center}
\end{figure}



\clearpage



\subsection{Punto 16}

\textbf{Enunciado}: \textsl{Hallar por simulación la variación de la tensión de salida en función del tiempo para un salto abrupto de la corriente de salida desde aproximadamente $0 \si[per-mode=symbol]{\ampere}$ hasta $ 1 \si[per-mode=symbol]{\ampere}$ y posteriormente un salto abrupto de la corriente de salida desde aproximadamente $ 1A$ hasta $0 A$. Considerar $R_{9} = 10 \si[per-mode=symbol]{\kilo\ohm}$ y $R_{18} = 0 \si[per-mode=symbol]{\ohm}$.}


\vspace{1.5cm}


Para lograr la conmutación de la carga se utilizó el circuito mostrado en la figura~\figref{fig:fig_p16_rl_switch}, donde se puede ver el modelo de una llave controlada por tensión con resistencia de $0 \si[per-mode=symbol]{\ohm}$ en estado cerrado y resistencia tendiendo a infinito (valor muy grande) para el estado abierto. El switch es controlado por una onda cuadrada, de tal manera de lograr una carga de $0 \si[per-mode=symbol]{\ampere}$ al comienzo de la simulación, de $1 \si[per-mode=symbol]{\ampere}$ ($R_{L} = 2 \si[per-mode=symbol]{\ohm}$) a los $20 \si[per-mode=symbol]{\milli\second}$, y luego nuevamente $0 \si[per-mode=symbol]{\ampere}$ a los $50 \si[per-mode=symbol]{\milli\second}$. La simulación realizada es del tipo transitorio (\textbf{SPICE} \textit{.tran}), la salida de la misma se muestra resaltando el momento de las transiciones en la figura~\figref{fig:fig_p16_output_in_load_jump}. En la figura~\figref{fig:fig_p16_output_variation_in_load_jump} se destaca la diferencia en la tensión de salida en carga respecto a en vacío, esta diferencia permite estimar la resistencia de salida de la fuente de alimentación, el valor obtenido es:\\


\begin{equation}
R_{o} = \frac{\Delta V}{\Delta I} = \frac{\num{3.67E-4} \si[per-mode=symbol]{\volt}}{1.0137 \si[per-mode=symbol]{\ampere}} = 642 \si[per-mode=symbol]{\micro\ohm}
\end{equation}

\begin{figure}[H] %htb
\begin{center}
\includegraphics[width=0.9 \textwidth, angle=0]{./img/preguntas/p16a.png}
\caption{\label{fig:fig_p16_rl_switch}\footnotesize{Circuito usado para la conmutación de la carga}}
\end{center}
\end{figure}


\vfill

\clearpage

\begin{figure}[H] %htb
\begin{center}
\includegraphics[width=1.2 \textwidth, angle=90]{./img/preguntas/p16b.png}
\caption{\label{fig:fig_p16_output_in_load_jump}\footnotesize{Tensión de salida frente a saltos de carga de $0 \si[per-mode=symbol]{\ampere}$ a $1 \si[per-mode=symbol]{\ampere}$ y de $1 \si[per-mode=symbol]{\ampere}$ a $0 \si[per-mode=symbol]{\ampere}$.}}
\end{center}
\end{figure}


\clearpage

\begin{figure}[H] %htb
\begin{center}
\includegraphics[width=1.2 \textwidth, angle=90]{./img/preguntas/p16c.png}
\caption{\label{fig:fig_p16_output_variation_in_load_jump}\footnotesize{Variación de la tensión de salida en los saltos de carga}}
\end{center}
\end{figure}




\clearpage


\subsection{Punto 17}

\textbf{Enunciado}: \textsl{Calcular la eficiencia para $V_{1}$ igual a $15 \si[per-mode=symbol]{\volt}$, $20 \si[per-mode=symbol]{\volt}$ y $25 \si[per-mode=symbol]{\volt}$.}

\begin{enumerate}
\item[\textsl{a)}] \textsl{con} $R_{L} = 10 \si[per-mode=symbol]{\ohm}$, $R_{9} = 90 \si[per-mode=symbol]{\kilo\ohm}$ y $R_{18} = 0 \si[per-mode=symbol]{\ohm}$.
\item[\textsl{b)}] \textsl{con} $R_{L} = 1 \si[per-mode=symbol]{\ohm}$, $R_{9} = 0 \si[per-mode=symbol]{\ohm}$ y $R_{18} = 0 \si[per-mode=symbol]{\ohm}$.
\end{enumerate}


\vspace{1.5cm}

Para calcular la eficiencia de la fuente de alimentación simplemente se aplicó la definición $\eta = \frac{P_{out}}{P_{in}}$, el cálculo se realizó en estado estacionario, ignorando el consumo durante los transitorios, los cuales de todas formas, a largo plazo, deben ser despreciables. El cálculo se realizó directamente dentro del \textbf{LTSPICE}, utilizando el comando de \textbf{SPICE}, \textit{.measure}, este comando permite realizar cálculos utilizando valores de variables simuladas, y luego operar con estos resultados. Realizando una simulación de \textbf{SPICE} de punto de operación, \textit{.op}, se realizan las siguientes mediciones:

%******************************************************************************************
\lstset{language=,xleftmargin=1em,numbers=none}

\lstset{showspaces=false}
\lstset{showstringspaces=false}
\normalfont
\normalsize
\lstset{backgroundcolor=\color{white},rulecolor=\color{blue}}
\lstset{basicstyle=\ttfamily\color{Deepblue}}

\lstset{keywordstyle=[1]\ttfamily\color{red}\bfseries}
\lstset{keywordstyle=[2]\ttfamily\color{LightSkyBlue}}
\lstset{keywordstyle=[3]\ttfamily\bfseries\color{Plum}}
\lstset{keywordstyle=[4]\ttfamily\bfseries\color{Chocolate}}

\lstset{identifierstyle=\ttfamily\color{black}}
\lstset{commentstyle=\ttfamily\color{blue}\textit}
\lstset{stringstyle=\ttfamily\color{purple}\upshape}
\lstset{tabsize=4}

\lstset{numberstyle=\ttfamily\color{Deeppurple}\upshape}
\lstset{numbersep=5pt}

\lstset{inputencoding=utf8/latin1}



\fontencoding{T1}
\fontseries{m}
\fontsize{10pt}{11pt}
\selectfont
%******************************************************************************************


\begin{lstlisting}
.op

.meas Pin PARAM V(Vin)*(-I(V1))
.meas Pout PARAM V(Vout)*(-I(RL))
.meas Eff PARAM Pout / Pin
\end{lstlisting}



Los resultados obtenidos para la eficiencia para cada uno de los valores de tensión de entrada para la que se realizó la simulación, se resumen en la tabla~\tableref{table:table_efficiency}. Se puede ver claramente como la eficiencia disminuye al aumentar la diferencia entre la tensión de entrada y la tensión de salida, cosa totalmente esperable, ya que, a tensión de salida constante, la tensión en el elemento de paso es cada vez mayor, a la misma corriente de carga, se tiene mayor potencia disipada en el mismo.\\
También la eficiencia empeora con el aumento de la carga, ya que al pasar la corriente de carga por el elemento de paso. se disipará también mas potencia en este.\\

%% \noindent
%% \begin{center}
 
%%\begin{spacing}{1}  
\bgroup
\begin{table}[H]  %%\centering
    
    \setlength\arrayrulewidth{1.5pt}
    \arrayrulecolor{white}
    \def\clinecolor{\hhline{|>{\arrayrulecolor{white}}-%
    >{\arrayrulecolor{white}}|-|-|-|-|}}
\resizebox{0.9 \textwidth}{!}{% 
       
\def\arraystretch{2.5}
\begin{tabularx}{1 \textwidth}%
    {|
    >{\columncolor{white} \centering\arraybackslash}m{0.25\linewidth}
     |
    >{\columncolor{white} \centering\arraybackslash}m{0.25\linewidth}
     |
    >{\columncolor{white} \centering\arraybackslash}m{0.25\linewidth}
     |
    >{\columncolor{white} \centering\arraybackslash}m{0.25\linewidth}
     |
    }
    \rowcolor{HeadersColor} \cellcolor{white} \thead{}  & \thead{$V_{i} = 15 \si[per-mode=symbol]{\volt}$} & \thead{$V_{i} = 20 \si[per-mode=symbol]{\volt}$} & \thead{$V_{i} = 25 \si[per-mode=symbol]{\volt}$} \\
    
    \hhline{|-|-|-|-|}
    \rowcolor{gray!20} \cellcolor{gray!60} \thead{ \cellcolor{gray!60} $\eta = \frac{P_{out}}{P_{in}} \cdot 100 \%$ \\ \cellcolor{gray!60} \\ \cellcolor{gray!60} $R_{L} = 10 \si[per-mode=symbol]{\ohm}$, $R_{9} = 90 \si[per-mode=symbol]{\kilo\ohm}$ \\ \cellcolor{gray!60} $R_{18} = 0 \si[per-mode=symbol]{\ohm}$ } & 63.93 \% & 48.04 \% & 38.50 \%  \\
    \hhline{|-|-|-|-|}
    \rowcolor{gray!20} \cellcolor{gray!60}  \thead{ \cellcolor{gray!60} $\eta = \frac{P_{out}}{P_{in}} \cdot 100 \%$ \\ \cellcolor{gray!60} \\ \cellcolor{gray!60} $R_{L} = 1 \si[per-mode=symbol]{\ohm}$, $R_{9} = 0 \si[per-mode=symbol]{\ohm}$ \\ \cellcolor{gray!60} $R_{18} = 0 \si[per-mode=symbol]{\ohm}$ } & 6.58 \% & 4.95 \% & 3.96 \%  \\
    \hhline{|-|-|-|-|}        
    \end{tabularx}}
	\caption{\footnotesize{Eficiencia en función de la tensión de entrada.}}
	\label{table:table_efficiency}
\end{table}
\egroup
%%\end{spacing}

%% \end{center}



\clearpage

\subsection{Punto 18}

\textbf{Enunciado}: \textsl{¿Cómo influye en la tensión de salida la variación de la fuente de entrada $V_{1}$ (variando de $1 \si[per-mode=symbol]{\volt}$ a $30 \si[per-mode=symbol]{\volt}$ y con $R_{L} = 10 \si[per-mode=symbol]{\ohm}$, $R_{9} = 90 \si[per-mode=symbol]{\kilo\ohm}$ y $R_{18} = 0 \si[per-mode=symbol]{\ohm}$)?. Simular para graficar la tensión de salida en función de $V_{1}$.}\\


Es de esperarse que la fuente tenga un valor mínimo de tensión de entrada, a partir del cual empieza a regular la salida, para valores menores de tensión que este valor mínimo, la salida será menor a la esperada, en principio, el regulador paralelo para regular a $10 V$, necesita una tensión de entrada mayor, pero también se deben polarizar correctamente los transistores, y como puede verse en la figura~\figref{fig:fig_p18_vo_vs_vi}, el crecimiento de la salida es prácticamente lineal, manteniendo una diferencia de aproximadamente $2.4V$ a la salida con respecto a la entrada, este valor sería el \quotemarks{drop-out} de esta fuente de alimentación. Tomando que la salida está regulando al 
llegar a aproximadamente al $1 \%$ de la tensión regulada esperada a la salida, $10 V$, el valor de tensión de entrada mínimo para salida regulada es de $12.38 V$. También puede observarse que para tensiones muy pequeñas a la entrada, hasta aproximadamente $1.8 V$, la tensión a la salida es prácticamente $0 V$, esto se explica por estar cortado el elemento de paso de la fuente de alimentación, el par compuesto.

\vfill

\clearpage

\begin{figure}[H] %htb
\begin{center}
\includegraphics[width=1.2 \textwidth, angle=90]{./img/preguntas/p18.png}
\caption{\label{fig:fig_p18_vo_vs_vi}\footnotesize{Tensión de salida vs tensión de entrada.}}
\end{center}
\end{figure}

\clearpage


\subsection{Punto 19}

\textbf{Enunciado}: \textsl{¿Cómo influye en la corriente de salida la variación de la fuente de entrada $V_{1}$ (variando de $1 \si[per-mode=symbol]{\volt}$ a $30 \si[per-mode=symbol]{\volt}$ y con $R_{L} = 0 \si[per-mode=symbol]{\ohm}$, $R_{9} = 90 \si[per-mode=symbol]{\kilo\ohm}$ y $R_{18} = 0 \si[per-mode=symbol]{\ohm}$?. Simular para graficar la corriente de salida en función de $V_{1}$.}\\

\vspace{1.5cm}

De la misma forma que el punto anterior, es de esperarse que la fuente tenga un valor mínimo de tensión de entrada, a partir del cual empieza a regular la salida, para valores menores de tensión que este valor mínimo, la corriente de salida será solo limitada a algún valor menor al esperado, y como puede verse en la figura~\figref{fig:fig_p19_io_vs_vi}, el crecimiento de la salida es prácticamente lineal, salvo que se produce un pico de corriente alrededor de los $3 V$ de entrada, que es limitado por la acción de $Q_{15}$, a ese valor de tensión de entrada el lazo de corriente seguramente no actúa de ninguna forma para limitar la corriente de salida. Tomando que la salida está regulando al llegar a aproximadamente al $1 \%$ de la corriente regulada esperada a la salida, $2.05 A$, el valor de tensión de entrada mínimo para salida regulada es de $12.19 V$, valor muy cercano al del punto anterior. También puede observarse que para tensiones muy pequeñas a la entrada, aproximadamente $1.8 V$, la corriente a la salida es prácticamente $0 A$, esto se explica por la misma razón que en el punto anterior.

\vfill

\clearpage

\begin{figure}[H] %htb
\begin{center}
\includegraphics[width=1.2 \textwidth, angle=90]{./img/preguntas/p19.png}
\caption{\label{fig:fig_p19_io_vs_vi}\footnotesize{Tensión de salida vs tensión de entrada.}}
\end{center}
\end{figure}

\clearpage


\subsection{Punto 20}

\textbf{Enunciado}: \textsl{Determinar el rechazo de ruido, o sea, ¿Cuántos decibles de diferencia se miden comparando un ruido presente en la tensión de entrada $V_{1}$ respecto del residuo de ese ruido en la tensión de salida. Debe intentarse no considerar el ruido propio de la fuente. \textbf{NOTA}: el ruido podría ser por ejemplo el rizado resultante de una rectificación y filtrado.}\\


\vspace{1.5cm}


Para ver el rechazo de ruido que presenta la fuente de alimentación, sumamos a la tensión de entrada una señal en forma de diente de sierra descendente de $100 Hz$, ya que la sugerencia era que el ruido podría provenir del rizado resultante de una rectificación y filtrado. Se utilizó una señal de $2 V$ de amplitud para poder apreciar bien la amplitud de la señal en la salida. Se graficó la entrada y la salida restándole la tensión continua de base, $20 V$ a la entrada y el valor mínimo a la salida, alrededor de $2 V$, se utilizó un script de \textbf{MATLAB} para restar los valores adecuados, hacer los cálculos y producir los gráficos a partir de los datos exportados del \textbf{LTSPICE}. En la figura~\figref{fig:fig_p20_ripple}, se muestra lo obtenido, mostrando simultáneamente 2 ciclos del rizado de entrada y salida.
Como se observa en la figura, la salida presenta un cierto tiempo de crecimiento debido al ancho de banda limitado del amplificador de la fuente, y presenta además un pequeño pico en la discontinuidad, el mismo se debe al tipo de compensación del circuito, tema que veremos en la siguiente parte del trabajo práctico. Si se mide el rechazo de ruido simplemente como el cociente de valor RMS de la señal de salida respecto de la entrada obtenemos:


\begin{equation}
    R_{nr}= 59.49 dB
\end{equation}

Sin embargo se pedía intentar no considerar el ruido propio de la fuente, entonces lo que se hizo fue, extrapolar la amplitud máxima de la señal de rizado a la salida, ignorando el sobre-pico y el tiempo de crecimiento, midiendo su amplitud cerca de la mitad de la amplitud máxima de la señal de entrada, calculado así, obtuvimos:


\begin{equation}
    R_{nr}= 61.41 dB
\end{equation}

\vfill

\clearpage

\begin{figure}[H] %htb
\begin{center}
\includegraphics[width=1.2 \textwidth, angle=90]{./img/preguntas/p20.png}
\caption{\label{fig:fig_p20_ripple}\footnotesize{Rizado de entrada y salida.}}
\end{center}
\end{figure}

\clearpage


\clearpage

\subsection{Punto 21}

Modificar el circuito de la fuente reemplazando en parte o totalmente el amplificador por el regulador integrado $LM723$ y evaluar el comportamiento del nuevo diseño comparándolo con el original.

\clearpage


