%Este archivo no tiene contenido, mas allá de configuraciones y\o definiciones.
%todo el contenido se encuentra en los archivos secundarios que son importados por este.

\documentclass[twoside,letterpaper]{article}


%\\\\\\\\\\\\\\\\\\\\\\\\\\\
%Packages en uso

%Idiomas diccionario
\usepackage[english, spanish]{babel}
%\usepackage[english]{babel}

\usepackage[utf8x]{inputenc}
%\usepackage[latin1]{inputenc} 
%\usepackage[ansinew]{inputenc}
\usepackage{ucs}




%\usepackage[margin=1cm, paperwidth=21.0cm, paperheight=29.6cm]{geometry}



\usepackage[T1]{fontenc} 
\usepackage{graphicx}
\usepackage{float}
\usepackage{longtable}
%\usepackage{floatflt}
\usepackage{fancyhdr}
\usepackage{hyperref}
%\usepackage{url}
\usepackage{amsfonts}
\usepackage{amssymb}
\usepackage{textcomp}
%\usepackage[symbol]{footmisc}
%\usepackage{pst-circ}
%\usepackage{epsfig}
%\usepackage{xkeyval}
\usepackage{tabularx}
\usepackage{booktabs}


%\usepackage{color}
\usepackage[usenames,dvipsnames]{color}


%\usepackage{minted}
\usepackage{latexsym}
\usepackage{colortbl}
%\usepackage{pdfpages}
\usepackage{wrapfig}

%\usepackage{listings}
\usepackage{listingsutf8}
%\usepackage{mips}
\usepackage{appendix}
\usepackage{needspace}
\usepackage{ifplatform}
\usepackage{ifthen}
\usepackage{amsmath}


%\\\\\\\\\\\\\\\\\\\\\\\\\\\
% FOR GNUPLOT
%\usepackage{tikz}

%\usepackage[miktex]{gnuplottex}
%\usepackage{gnuplot-lua-tikz}
%\usepackage{mathpazo}
%\\\\\\\\\\\\\\\\\\\\\\\\\\\


%\\\\\\\\\\\\\\\\\\\\\\\\\\\
% FOR FIGURE CAPTION COLORS

\usepackage{caption}
\usepackage[svgnames]{xcolor}

%\\\\\\\\\\\\\\\\\\\\\\\\\\\


%\\\\\\\\\\\\\\\\\\\\\\\\\\\
% FOR EQUATION CAPTION FORMAT, COLORS AND OTHERS

\usepackage{mathtools}

%\\\\\\\\\\\\\\\\\\\\\\\\\\\



%\\\\\\\\\\\\\\\\\\\\\\\\\\\
%!!!!!!!!!!!!!!!!!!!!!!!!! DETECCION DE PLATAFORMA !!!!!!!!!!!!!!!!!!!!!!!!!!!!!!!!!
%Permite que sea compilado tanto en windows como en *IX sin cambio alguno.
\newboolean{IsWindows}
\ifwindows
\setboolean{IsWindows}{true}
\else
\setboolean{IsWindows}{false}
\fi
%\\\\\\\\\\\\\\\\\\\\\\\\\\\
%!!!!!!!!!!!!!!!!!!!!!!!!!!!!!!!!!!!!!!!!!!!!!!!!!!!!!!!!!!!!!!!!!!!!!!!!!!!!!!!!!!!!

%\\\\\\\\\\\\\\\\\\\\\\\\\\\
%Idiomas
\hyphenrules{spanish}
%\\\\\\\\\\\\\\\\\\\\\\\\\\\






%\\\\\\\\\\\\\\\\\\\\\\\\\\\
%Comandos personalizados

\newcommand{\titulo}{Trabajo práctico N\textdegree \hspace{1pt} 1A}
\newcommand{\titulolargo}{Análisis de fuente lineal}
\newcommand{\materia}{Circuitos electrónicos II - 66.10}
\newcommand{\fiuba}{Facultad de Ingeniería - UBA}
\newcommand{\cuatrimestre}{1\textsuperscript{er} cuatrimestre de 2019} %\sptext{do} $2^{do}$


\newcommand{\autorA}{\textsc{Irusta} Pablo}
\newcommand{\padronA}{80171}
\newcommand{\mailA}{\weblink{mailto:pabirus@gmail.com}{pabirus@gmail.com}}


\newcommand{\autorB}{\textsc{Luna} Diego}
\newcommand{\padronB}{75451}
\newcommand{\mailB}{\weblink{mailto:diegorluna@gmail.com}{diegorluna@gmail.com}} 
 
\newcommand{\autorC}{\textsc{Niero} Adrián}
\newcommand{\padronC}{80533}
\newcommand{\mailC}{\weblink{mailto:adrianniero@gmail.com}{adrianniero@gmail.com}} 
 
 
 
\newcommand{\docenteA}{Ing. \textsc{Bertuccio} José Alberto}
\newcommand{\docenteB}{Ing. \textsc{Acquaticci} Fabián}
\newcommand{\docenteC}{Ing. \textsc{Marchi} Edgardo }
\newcommand{\docenteD}{Ing. \textsc{Bulacio} Matías}
\newcommand{\docenteE}{Ing. \textsc{D’Angiolo} Federico}
\newcommand{\docenteF}{Ing. \textsc{Gamez} Pablo}


\newcommand{\thedate}{13 de abril de 2019}


\newcommand{\HRule}{\rule{\linewidth}{0.3mm}}

%\\\\\\\\\\\\\\\\\\\\\\\\\\\


%\\\\\\\\\\\\\\\\\\\\\\\\\\\
%Título,  autor del documento y fecha
\title{\titulo}
\author{\autorB}
\date{\thedate}
%\\\\\\\\\\\\\\\\\\\\\\\\\\\


%\\\\\\\\\\\\\\\\\\\\\\\\\\\
\setcounter{secnumdepth}{5}
\setcounter{tocdepth}{5}
%\\\\\\\\\\\\\\\\\\\\\\\\\\\


%\\\\\\\\\\\\\\\\\\\\\\\\\\\
%suppress widows and orphans
\widowpenalty=9999
\clubpenalty=9999
%\\\\\\\\\\\\\\\\\\\\\\\\\\\


%\\\\\\\\\\\\\\\\\\\\\\\\\\\
%equation numbers to subsection level
%\numberwithin{equation}{subsection}
\numberwithin{equation}{section}
%\\\\\\\\\\\\\\\\\\\\\\\\\\\


%\\\\\\\\\\\\\\\\\\\\\\\\\\\
%equation numbers to subsection level
%\numberwithin{table}{subsection}
\numberwithin{table}{section}
%\\\\\\\\\\\\\\\\\\\\\\\\\\\



%\\\\\\\\\\\\\\\\\\\\\\\\\\\
%figure numbers to subsection level
%\renewcommand{\thefigure}{\thesubsection.\arabic{figure}}
\renewcommand{\thefigure}{\thesection.\arabic{figure}}
%\\\\\\\\\\\\\\\\\\\\\\\\\\\

%\setlength{\arrayrulewidth}{0.6pt}


%\\\\\\\\\\\\\\\\\\\\\\\\\\\
\newcolumntype{z}[1]{%
>{\centering\hspace{0pt}}p{#1}}%

\newcolumntype{y}[1]{%
>{\raggedleft\hspace{0pt}}p{#1}}% 

\newcolumntype{x}[1]{%
>{\raggedright\hspace{0pt}}p{#1}}% 

\newcolumntype{w}[1]{%
>{\centering\hspace{0pt}}m{#1}}%

\newcolumntype{v}[1]{%
>{\raggedleft\hspace{0pt}}m{#1}}% 

\newcolumntype{u}[1]{%
>{\raggedright\hspace{0pt}}m{#1}}% 


\newcommand{\tn}{\tabularnewline}
%\\\\\\\\\\\\\\\\\\\\\\\\\\\




%\\\\\\\\\\\\\\\\\\\\\\\\\\\
% Generales
\newcommand{\quotemarks}[1]{``#1''}
\newcommand{\simplequotemarks}[1]{`#1'}
%\\\\\\\\\\\\\\\\\\\\\\\\\\\


%\\\\\\\\\\\\\\\\\\\\\\\\\\\
% Símbolos de las unidades

\newcommand{\volt}[1]{\mbox{#1 V}}
\newcommand{\milivolt}[1]{\mbox{#1 mV}}
\newcommand{\hertz}[1]{\mbox{#1 Hz}}
\newcommand{\kilohertz}[1]{\mbox{#1 kHz}}
\newcommand{\megahertz}[1]{\mbox{#1 MHz}}
\newcommand{\farad}[1]{\mbox{#1 F}}
\newcommand{\nanofarad}[1]{\mbox{#1 nF}}
\newcommand{\microfarad}[1]{\mbox{#1 $\mu$F}}
\newcommand{\picofarad}[1]{\mbox{#1 pF}}
\newcommand{\fentofarad}[1]{\mbox{#1 fF}}
\newcommand{\ohm}[1]{\mbox{#1 $\Omega$}}
\newcommand{\miliohm}[1]{\mbox{#1 m$\Omega$}}
\newcommand{\kiloohm}[1]{\mbox{#1 k$\Omega$}}
\newcommand{\megaohm}[1]{\mbox{#1 M$\Omega$}}
\newcommand{\amper}[1]{\mbox{#1 A}}
\newcommand{\miliamper}[1]{\mbox{#1 mA}}
\newcommand{\microamper}[1]{\mbox{#1 $\mu$A}}
\newcommand{\picoamper}[1]{\mbox{#1 pA}}
\newcommand{\fentoamper}[1]{\mbox{#1 fA}}
\newcommand{\s}[1]{\mbox{#1 s}}
\newcommand{\milis}[1]{\mbox{#1 ms}}
\newcommand{\micros}[1]{\mbox{#1 $\mu$s}}
\newcommand{\nanos}[1]{\mbox{#1 ns}}
\newcommand{\miliamperporvolt}[1]{ \mbox{#1 $\frac{mA}{V}$}}
\newcommand{\miliamperporvoltcuad}[1]{ \mbox{#1 $\frac{mA}{V^2}$}}
\newcommand{\decibel}[1]{\mbox{#1 dB}}
\newcommand{\decibeli}[1]{\mbox{#1 dBi}}

\newcommand{\spice}{\mbox{\textit{\textbf{SPICE}}}}
\newcommand{\schematic}{\mbox{\textit{\textbf{SCHEMATIC}}}}

% Nombres de las unidades
\newcommand{\Metro}{\mbox{metro}}
\newcommand{\Volt}{\mbox{Volt}}
\newcommand{\Amper}{\mbox{Ampere}}
\newcommand{\Farad}{\mbox{Farad}}
%\\\\\\\\\\\\\\\\\\\\\\\\\\\

%\\\\\\\\\\\\\\\\\\\\\\\\\\\
% Dispositivos

\newcommand{\mosfet}{\mbox{\textbf{MOSFET}}}
\newcommand{\nmosfet}{\mbox{\textbf{NMOSFET}}}
\newcommand{\bjtnpn}{\mbox{\textbf{BJT NPN}}}
\newcommand{\bjtpnp}{\mbox{\textbf{BJT PNP}}}

%\\\\\\\\\\\\\\\\\\\\\\\\\\\

%\\\\\\\\\\\\\\\\\\\\\\\\\\\
% Plataformas

\newcommand{\platformhost}{\textbf{x86(\_64)\textbackslash Linux}}
\newcommand{\platformguest}{\textbf{pmax\textbackslash NetBSD}}

\newcommand{\oshost}{\textbf{Linux}}
\newcommand{\osguest}{\textbf{NetBSD}}

\newcommand{\MIPS}{\textbf{MIPS32}}
%\\\\\\\\\\\\\\\\\\\\\\\\\\\


%\\\\\\\\\\\\\\\\\\\\\\\\\\\
% Programación

\newcommand{\GNU}{\textbf{GNU}}

\newcommand{\GCC}{\textbf{GCC}}

\newcommand{\GDB}{\textbf{GDB}}

\newcommand{\GXEMUL}{\textbf{GXemul}}

\newcommand{\langc}{\textbf{\quotemarks{C}}}

\newcommand{\langass}{\textbf{assembly}}

\newcommand{\langmipsass}{\textbf{MIPS32 assembly}}

\newcommand{\make}{\textbf{make}}
%\\\\\\\\\\\\\\\\\\\\\\\\\\\


%\\\\\\\\\\\\\\\\\\\\\\\\\\\
% Archivos

\newcommand{\quotefile}[1]{\textit{\quotemarks{#1}}}

\newcommand{\filebox}[2]{%

\begin{tabular}{l}

\multicolumn{1}{>{\columncolor{#2}}l}{#1} 
	
\end{tabular}

}%
%\\\\\\\\\\\\\\\\\\\\\\\\\\\



%\\\\\\\\\\\\\\\\\\\\\\\\\\\
% Math
\newcommand{\Reales}{\mathbb{R}}
\newcommand{\Complejos}{\mathbb{C}}
\newcommand{\numnorm}[1]{\left|#1\right|}
\newcommand{\vectornorm}[1]{\left|\left|#1\right|\right|}
%\\\\\\\\\\\\\\\\\\\\\\\\\\\


%\\\\\\\\\\\\\\\\\\\\\\\\\\\
% Counters
\newcommand{\resetallcounters}{%
\setcounter{figure}{0}
\setcounter{equation}{0}
\setcounter{table}{0}
}%
%\\\\\\\\\\\\\\\\\\\\\\\\\\\



%\\\\\\\\\\\\\\\\\\\\\\\\\\\
% Definiciones de colores.
\definecolor{Deepblue}{rgb}{0.00,0.00,0.70}
\definecolor{Deepgreen}{rgb}{0.09,0.45,0.20}
\definecolor{Darkgreen}{RGB}{0, 128, 0}
\definecolor{Purple}{rgb}{1,0,1}
\definecolor{Deeppurple}{rgb}{0.2,0,1}
\definecolor{Gray}{rgb}{0.3,0.3,0.3}
\definecolor{Lightblue}{rgb}{0.60, 0.80, 1.00}
\definecolor{Lightyellow}{rgb}{1.00,1.00,0.60}
\definecolor{LightButter}{rgb}{0.98,0.91,0.31}
\definecolor{LightOrange}{rgb}{0.98,0.68,0.24}
\definecolor{LightChocolate}{rgb}{0.91,0.72,0.43}
\definecolor{LightChameleon}{rgb}{0.54,0.88,0.20}
\definecolor{LightSkyBlue}{rgb}{0.45,0.62,0.81}
\definecolor{LightPlum}{rgb}{0.68,0.50,0.66}
\definecolor{LightScarletRed}{rgb}{0.93,0.16,0.16}
\definecolor{Butter}{rgb}{0.93,0.86,0.25}
\definecolor{Orange}{rgb}{0.96,0.47,0.00}
\definecolor{Chocolate}{rgb}{0.75,0.49,0.07}
\definecolor{Chameleon}{rgb}{0.45,0.82,0.09}
\definecolor{SkyBlue}{rgb}{0.20,0.39,0.64}
\definecolor{Plum}{rgb}{0.46,0.31,0.48}
\definecolor{ScarletRed}{rgb}{0.80,0.00,0.00}
\definecolor{DarkButter}{rgb}{0.77,0.62,0.00}
\definecolor{DarkOrange}{rgb}{0.80,0.36,0.00}
\definecolor{DarkChocolate}{rgb}{0.56,0.35,0.01}
\definecolor{DarkChameleon}{rgb}{0.30,0.60,0.02}
\definecolor{DarkSkyBlue}{rgb}{0.12,0.29,0.53}
\definecolor{DarkPlum}{rgb}{0.36,0.21,0.40}
\definecolor{DarkScarletRed}{rgb}{0.64,0.00,0.00}
\definecolor{Aluminium1}{rgb}{0.93,0.93,0.92}
\definecolor{Aluminium2}{rgb}{0.82,0.84,0.81}
\definecolor{Aluminium3}{rgb}{0.73,0.74,0.71}
\definecolor{Aluminium4}{rgb}{0.53,0.54,0.52}
\definecolor{Aluminium5}{rgb}{0.33,0.34,0.32}
\definecolor{Aluminium6}{rgb}{0.18,0.20,0.21}

\definecolor{EQColor}{rgb}{0.00,0.26,0.36} % {0.80,0.36,0.00}
\definecolor{FIGColor}{cmyk}{1,0.00,0.00,0.00}
\definecolor{TABLEColor}{RGB}{0,199,199}



\definecolor{APENDLINKColor}{rgb}{0.96,0.47,0.00}
\definecolor{SECTLINKColor}{rgb}{1.00,0.00,0.00}
\definecolor{FILELINKColor}{rgb}{1.00,0.00,0.00}
\definecolor{INTERNALLINKColor}{rgb}{1.00,0.00,0.00}
\definecolor{WEBLINKColor}{rgb}{0.00,0.00,1.00}
\definecolor{CITELINKColor}{RGB}{141,199,126}
\definecolor{TABLELINKColor}{RGB}{199,199,0}


%\definecolor{EQColor}{rgb}{0.00,0.36,0.80}
%\definecolor{FIGColor}{cmyk}{1,0.00,0.00,0.00}
%\definecolor{TABLEColor}{RGB}{0,199,25}



%\definecolor{APENDLINKColor}{rgb}{0.10,0.47,0.00}
%\definecolor{SECTLINKColor}{rgb}{0.00,0.00,1.00}
%\definecolor{FILELINKColor}{rgb}{0.00,0.00,1.00}
%\definecolor{INTERNALLINKColor}{rgb}{0.00,0.00,1.00}
%\definecolor{WEBLINKColor}{rgb}{0.00,0.00,1.00}
%\definecolor{CITELINKColor}{RGB}{141,199,126}
%\definecolor{TABLELINKColor}{RGB}{199,199,0}


%\\\\\\\\\\\\\\\\\\\\\\\\\\\


%\\\\\\\\\\\\\\\\\\\\\\\\\\\
% FOR FIGURE CAPTION COLORS

\DeclareCaptionFont{FIGFont}{\color{FIGColor}}
\captionsetup[figure]{labelfont={FIGFont,bf}}

\newcommand{\figref}[1]{\textcolor{FIGColor}{[\ref{#1}]}}
%\\\\\\\\\\\\\\\\\\\\\\\\\\\


%\\\\\\\\\\\\\\\\\\\\\\\\\\\
% FOR TABLE CAPTION COLORS

\DeclareCaptionFont{TABLEFont}{\color{TABLEColor}}
\captionsetup[table]{labelfont={TABLEFont,bf}}

\newcommand{\tableref}[1]{\textcolor{TABLEColor}{[\ref{#1}]}}


%\captionsetup[table]{style=fortables}
%\captionsetup[figure]{style=forfigures}
%\\\\\\\\\\\\\\\\\\\\\\\\\\\


%\\\\\\\\\\\\\\\\\\\\\\\\\\\
% FOR EQUATION CAPTION FORMAT, COLORS AND OTHERS

\newtagform{brackets2}[\textcolor{EQColor}]{\textcolor{EQColor}(}{\textcolor{EQColor})}
\usetagform{brackets2}

%\\\\\\\\\\\\\\\\\\\\\\\\\\\


%\\\\\\\\\\\\\\\\\\\\\\\\\\\
% FOR EQUATION CAPTION COLORS
%\makeatletter %% Without ams
%\def\@eqnnum{{\normalfont\normalcolor[\theequation]}}
%\makeatother

%But amsmath redefines the numbering of equations, so then you can do:

%\makeatletter %% With ams
%\def\tagform@#1{\maketag@@@{[\ignorespaces#1\unskip\@@italiccorr]}}
%\makeatother 
%\\\\\\\\\\\\\\\\\\\\\\\\\\\


%\\\\\\\\\\\\\\\\\\\\\\\\\\\
% FOR APENDIX REFERENCE

\newcommand{\apendref}[1]{\textcolor{APENDLINKColor}{[\ref{#1}]}}

%\\\\\\\\\\\\\\\\\\\\\\\\\\\


%\\\\\\\\\\\\\\\\\\\\\\\\\\\
% FOR SECTION REFERENCE

\newcommand{\sectref}[1]{\textcolor{SECTLINKColor}{[\ref{#1}]}}

%\\\\\\\\\\\\\\\\\\\\\\\\\\\


%\\\\\\\\\\\\\\\\\\\\\\\\\\\
% WEB LINK

\newcommand{\weblink}[2]{\href{#1}{\textcolor{WEBLINKColor}{#2}}}

%\\\\\\\\\\\\\\\\\\\\\\\\\\\


%\\\\\\\\\\\\\\\\\\\\\\\\\\\
% FILE LINK

\newcommand{\filelink}[2]{\href{#1}{\textcolor{FILELINKColor}{#2}}}

%\\\\\\\\\\\\\\\\\\\\\\\\\\\


%\\\\\\\\\\\\\\\\\\\\\\\\\\\
% CITE LINK
\newcommand{\citelink}[1]{\textcolor{LimeGreen}{\cite{#1}}}

%\\\\\\\\\\\\\\\\\\\\\\\\\\\


\hypersetup{
%	bookmarksnumbered,
%	pdfpagemode={UseOutlines},
%    bookmarks=true,         				% show bookmarks bar?
    unicode=true,          					% non-Latin characters in Acrobat’s bookmarks
    pdftoolbar=true,        				% show Acrobat’s toolbar?
    pdfmenubar=true,        				% show Acrobat’s menu?
%    pdffitwindow=false,    				% window fit to page when opened
%    pdfstartview={FitH},   				% fits the width of the page to the window
    pdftitle={\titulo},   					% title
    pdfauthor={\autorA},    				% author
    pdfsubject={\materia},   				% subject of the document
%    pdfcreator={\LaTeX},   				% creator of the document
%    pdfproducer={Producer},				% producer of the document
    pdfkeywords={TL} {TP}, 					% list of keywords
    pdfnewwindow=true,      				% links in new window
    linktoc=all,							%
    colorlinks=false,  						% false: boxed links; true: colored links	
	linkcolor=INTERNALLINKColor,			% color of internal links
    citecolor=CITELINKColor,    			% color of links to bibliography
    filecolor=FILELINKColor,   	 			% color of file links
    urlcolor=WEBLINKColor,       			% color of external links	
	linkbordercolor=INTERNALLINKColor,		% color of internal links
	citebordercolor=CITELINKColor,			% color of links to bibliography
    filebordercolor=FILELINKColor,    		% color of file links
    urlbordercolor=WEBLINKColor}	       	% color of external links	    
   				
	


%\\\\\\\\\\\\\\\\\\\\\\\\\\\
%Comportamiento de los links a archivos externos.
%\hypersetup{pdfnewwindow=true}
%\\\\\\\\\\\\\\\\\\\\\\\\\\\

%\\\\\\\\\\\\\\\\\\\\\\\\\\\
%Tamaños de la página y margenes

\linespread{1.3}
\oddsidemargin .1cm
\evensidemargin .1cm
\textwidth 16.5cm
\topmargin 0in
\voffset = 0pt
\textheight 21.08cm %8.3in
%\\\\\\\\\\\\\\\\\\\\\\\\\\\



%\pagestyle{fancy}


%\\\\\\\\\\\\\\\\\\\\\\\\\\\
%Encabezado y pie de páginas para todas las páginas normales

\fancypagestyle{allpages}{%

\fancyhf{}  

\renewcommand{\headrulewidth}{0.4pt}
\renewcommand{\footrulewidth}{0.4pt}

%No convierte a mayúscula los nombres de capítulo y sección 
%ni muestra el número.
%\renewcommand{\chaptermark}[1]{}
\renewcommand{\sectionmark}[1]{\markright{##1}{}}
\renewcommand{\subsectionmark}[1]{}
\renewcommand{\subsubsectionmark}[1]{}


\setlength{\headheight}{12pt}

   
%\fancyhf[LOH,REH]{\fiuba}
%\fancyhf[ROH,LEH]{\titulo}

\fancyhf[LH]{\small{\fiuba}}
\fancyhf[CH]{\small{\materia}}
\fancyhf[RH]{\titulo} 

\fancyhf[LOF,REF]{\small{\cuatrimestre}}
\fancyhf[CF]{\small{\rightmark}}
\fancyhf[LEF, ROF]{\textbf{\thepage}} 

}

%\\\\\\\\\\\\\\\\\\\\\\\\\\\


%\\\\\\\\\\\\\\\\\\\\\\\\\\\
%Encabezado y pie de páginas para el índice
\fancypagestyle{indexstyle}{%

\fancyhf{} % clear all header and footer fields

\renewcommand{\headrulewidth}{0.4pt}
\renewcommand{\footrulewidth}{0.4pt}

\setlength{\headheight}{12pt}


\fancyhf[LH]{\small{\fiuba}}
\fancyhf[CH]{\small{\materia}}
\fancyhf[RH]{\titulo}

\fancyhf[LOF,REF]{\small{\cuatrimestre}}
\fancyhf[CF]{\small{Índice}}
\fancyhf[LEF, ROF]{\textbf{\thepage}} 

%\color{red}

}
%\\\\\\\\\\\\\\\\\\\\\\\\\\\




%\\\\\\\\\\\\\\\\\\\\\\\\\\\
%Encabezado y pie de páginas para el código
\fancypagestyle{codestyle}{%

\fancyhf{} % clear all header and footer fields

\renewcommand{\headrulewidth}{0.4pt}
\renewcommand{\footrulewidth}{0.4pt}

%No convierte a mayúscula los nombres de capítulo y sección 
%ni muestra el número.
%\renewcommand{\chaptermark}[1]{}
\renewcommand{\sectionmark}[1]{}
\renewcommand{\subsectionmark}[1]{}
\renewcommand{\subsubsectionmark}[1]{\markright{##1}}

\setlength{\headheight}{12pt}

\fancyhf[LH]{\small{\fiuba}}
\fancyhf[CH]{\small{\materia}}
\fancyhf[RH]{\titulo}
\fancyhf[LOF,REF]{\small{\cuatrimestre}}
\fancyhf[CF]{\small{Archivo: \color{DarkBlue}\rightmark}}
%\bfseries{\color{red} \filename
\fancyhf[LEF, ROF]{\textbf{\thepage}} 


\normalfont
}
%\\\\\\\\\\\\\\\\\\\\\\\\\\\



\newcommand{\themark}{}

%\\\\\\\\\\\\\\\\\\\\\\\\\\\
%Encabezado y pie de páginas para el código
\fancypagestyle{codeconsstyle}{%

\fancyhf{} % clear all header and footer fields

\renewcommand{\headrulewidth}{0.4pt}
\renewcommand{\footrulewidth}{0.4pt}


\setlength{\headheight}{12pt}


\fancyhf[LH]{\small{\fiuba}}
\fancyhf[CH]{\small{\materia}}
\fancyhf[RH]{\titulo}

\fancyhf[LOF,REF]{\small{\cuatrimestre}}
\fancyhf[CF]{\small{\themark}}
\fancyhf[LEF, ROF]{\textbf{\thepage}} 

}
%\\\\\\\\\\\\\\\\\\\\\\\\\\\




%\\\\\\\\\\\\\\\\\\\\\\\\\\\
%Encabezado y pie de páginas para el índice
\fancypagestyle{bibliostyle}{%

\fancyhf{} % clear all header and footer fields

\renewcommand{\headrulewidth}{0.4pt}
\renewcommand{\footrulewidth}{0.4pt}

\setlength{\headheight}{12pt}


\fancyhf[LH]{\small{\fiuba}}
\fancyhf[CH]{\small{\materia}}
\fancyhf[RH]{\titulo}

\fancyhf[LOF,REF]{\small{\cuatrimestre}}
\fancyhf[CF]{\small{Bibliografía}}
\fancyhf[LEF, ROF]{\textbf{\thepage}} 

}
%\\\\\\\\\\\\\\\\\\\\\\\\\\\



%\\\\\\\\\\\\\\\\\\\\\\\\\\\
%Renuevo los nombres de los apéndices.
\renewcommand{\appendixpagename}{Apéndices}
\renewcommand{\appendixtocname}{Apéndices}
%\\\\\\\\\\\\\\\\\\\\\\\\\\\

%\\\\\\\\\\\\\\\\\\\\\\\\\\\
%Renuevo el símbolo de los items.
\renewcommand\labelitemi{$\bullet$}
%\\\\\\\\\\\\\\\\\\\\\\\\\\\




%\\\\\\\\\\\\\\\\\\\\\\\\\\\\\\\