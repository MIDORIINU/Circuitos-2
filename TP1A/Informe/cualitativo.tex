
\subsection{Secciones del circuito}


\normalfont

La topología del circuito corresponde a la de un típico amplificador de potencia de tres etapas realimentado, donde la \quotemarks{señal} a amplificar es una referencia de tensión, armada en torno a una referencia de tensión comercial, el \textbf{TL431}, la tensión de salida es muestreada y sumada a la entrada, formando un lazo de realimentación \textbf{serie-paralelo}, estabilizando la tensión de salida, el resultado de esta configuración es una fuente de tensión regulada. El circuito además posee un segundo lazo de realimentación, donde se muestrea la corriente de salida, se convierte a tensión y se suma a la entrada, formando un lazo de realimentación \textbf{serie-serie}, estabilizando la corriente de salida. El circuito trabaja con solo uno de los lazos de realimentación funcionando en un dado momento, el switcheo de uno a otro, se realiza en forma automática, con un subcircuito dedicado, según sea el estado de carga, el amplificador de potencia es el mismo en ambos lazos, solo cambia la red de realimentación. El circuito además cuenta con una limitación extra de corriente que actúa únicamente durante transitorios, además el circuito se encuentra compensado en frecuencia en ambos lazos (tema de la segunda parte del trabajo práctico).
En el circuito se pueden diferenciar claramente las secciones que se marcan en la figura~\figref{fig:fig_complete_circuit_secions}, las mismas son:


\begin{itemize}
\item Amplificador diferencial con caga activa: realiza la suma (resta) de la señal realimentada y provee amplificación.
\item Referencia de tensión: Provee una tensión estable de referencia de aproximadamente $1 V$ y además provee alimentación para algunas partes del circuito ($10 V$).
\item Seguidor con carga activa: Provee adaptación de impedancia entre la primera y la tercera etapa.
\item Par compuesto (Sziklai): Maneja la corriente de salida, presentando a la carga una muy baja impedancia y una alta impedancia a la segunda etapa.
\item Limitación de corriente simple: Formada solo por un transistor que limita durante transitorios, simplemente deriva corriente de la base del seguidor (segunda etapa).
\item Llave analógica: Hace el switcheo automático entre los lazos de tensión y corriente, es prácticamente transparente a fines prácticos.
\item Realimentación de tensión: Red de muestreo y realimentación de tensión (la mitad de la llave forma parte de la misma).
\item Realimentación de corriente: Red de muestreo y realimentación de corriente (la mitad de la llave forma parte de la misma).
\end{itemize}

