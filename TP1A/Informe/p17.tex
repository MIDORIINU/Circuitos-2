\vspace{1.5cm}

Para calcular la eficiencia de la fuente de alimentación simplemente se aplicó la definición $\eta = \frac{P_{out}}{P_{in}}$, el cálculo se realizó en estado estacionario, ignorando el consumo durante los transitorios, los cuales de todas formas, a largo plazo, deben ser despreciables. El cálculo se realizó directamente dentro del \textbf{LTSPICE}, utilizando el comando de \textbf{SPICE}, \textit{.measure}, este comando permite realizar cálculos utilizando valores de variables simuladas, y luego operar con estos resultados. Realizando una simulación de \textbf{SPICE} de punto de operación, \textit{.op}, se realizan las siguientes mediciones:

%******************************************************************************************
\lstset{language=,xleftmargin=1em,numbers=none}

\lstset{showspaces=false}
\lstset{showstringspaces=false}
\normalfont
\normalsize
\lstset{backgroundcolor=\color{white},rulecolor=\color{blue}}
\lstset{basicstyle=\ttfamily\color{Deepblue}}

\lstset{keywordstyle=[1]\ttfamily\color{red}\bfseries}
\lstset{keywordstyle=[2]\ttfamily\color{LightSkyBlue}}
\lstset{keywordstyle=[3]\ttfamily\bfseries\color{Plum}}
\lstset{keywordstyle=[4]\ttfamily\bfseries\color{Chocolate}}

\lstset{identifierstyle=\ttfamily\color{black}}
\lstset{commentstyle=\ttfamily\color{blue}\textit}
\lstset{stringstyle=\ttfamily\color{purple}\upshape}
\lstset{tabsize=4}

\lstset{numberstyle=\ttfamily\color{Deeppurple}\upshape}
\lstset{numbersep=5pt}

\lstset{inputencoding=utf8/latin1}



\fontencoding{T1}
\fontseries{m}
\fontsize{10pt}{11pt}
\selectfont
%******************************************************************************************


\begin{lstlisting}
.op

.meas Pin PARAM V(Vin)*(-I(V1))
.meas Pout PARAM V(Vout)*(-I(RL))
.meas Eff PARAM Pout / Pin
\end{lstlisting}



Los resultados obtenidos para la eficiencia para cada uno de los valores de tensión de entrada para la que se realizó la simulación, se resumen en la tabla~\tableref{table:table_efficiency}. Se puede ver claramente como la eficiencia disminuye al aumentar la tensión de entrada, cosa totalmente esperable, ya que, a tensión de salida constante, la tensión en el elemento de paso es cada vez mayor, a la misma corriente de carga, se tiene mayor potencia disipada en el mismo.


%% \noindent
%% \begin{center}
 
%%\begin{spacing}{1}  
\bgroup
\begin{table}[H]  %%\centering
    
    \setlength\arrayrulewidth{1.5pt}
    \arrayrulecolor{white}
    \def\clinecolor{\hhline{|>{\arrayrulecolor{white}}-%
    >{\arrayrulecolor{white}}|-|-|-|-|}}
\resizebox{0.9 \textwidth}{!}{% 
       
\def\arraystretch{2.5}
\begin{tabularx}{1 \textwidth}%
    {|
    >{\columncolor{white} \centering\arraybackslash}m{0.25\linewidth}
     |
    >{\columncolor{white} \centering\arraybackslash}m{0.25\linewidth}
     |
    >{\columncolor{white} \centering\arraybackslash}m{0.25\linewidth}
     |
    >{\columncolor{white} \centering\arraybackslash}m{0.25\linewidth}
     |
    }
    \rowcolor{HeadersColor} \cellcolor{white} \thead{}  & \thead{$V_{i} = 15 \si[per-mode=symbol]{\volt}$} & \thead{$V_{i} = 20 \si[per-mode=symbol]{\volt}$} & \thead{$V_{i} = 25 \si[per-mode=symbol]{\volt}$} \\
    
    \hhline{|-|-|-|-|}
    \rowcolor{gray!20} \cellcolor{gray!40} $\eta = \frac{P_{out}}{P_{in}}$ & 0.639 & 0.480 & 0.385  \\
    \hhline{|-|-|-|-|}       
    \end{tabularx}}
	\caption{\footnotesize{Eficiencia en función de la tensión de entrada.}}
	\label{table:table_efficiency}
\end{table}
\egroup
%%\end{spacing}

%% \end{center}












\clearpage