%*********************************************
\subsection{Consideraciones para los archivos de salida (archivos \textbf{\quotemarks{.out})}}
\thispagestyle{codeconsstyle}
\renewcommand{\themark}{Consideraciones para los archivos de salida (archivos \textbf{\quotemarks{.out}})}

El siguiente apéndice contiene los archivos \textbf{\quotemarks{.out}} completos producidos por las simulaciones en \spice, los mismos son transcritos completos y fueron producidos indicando que se incluya información detallada de polarización de los dispositivos semiconductores. La presentación de los archivos en el informe se hizo directamente desde los mismos hacia \LaTeX\space usando el paquete \textbf{\quotemarks{listingsutf8}}~\citelink{The_ListingsUTF8_Package}, que es una extensión con soporte para \textbf{UTF8} del paquete \textbf{\quotemarks{listings}}~\citelink{The_Listings_Package}, este paquete produce una salida formateada y con coloreado del contenido de los archivos y también permite el agregado de números de líneas y otros agregados. 
 

\clearpage
%*********************************************

%******************************************************************************************
\lstset{language=,numbers=left,xleftmargin=1em,stepnumber=1}

\lstset{showspaces=false}
\lstset{showstringspaces=false}

\lstset{backgroundcolor=\color{white},rulecolor=\color{blue}}
\lstset{basicstyle=\ttfamily\color{DarkBlue}}

%\lstset{keywordstyle=[1]\ttfamily\color{red}\bfseries}
%\lstset{keywordstyle=[2]\ttfamily\color{LightSkyBlue}}
%\lstset{keywordstyle=[3]\ttfamily\bfseries\color{Plum}}
%\lstset{keywordstyle=[4]\ttfamily\bfseries\color{Chocolate}}

%\lstset{identifierstyle=\ttfamily\color{black}}
%\lstset{commentstyle=\ttfamily\color{Aluminium4}\textit}
%\lstset{stringstyle=\ttfamily\color{purple}\upshape}
\lstset{tabsize=4}

\lstset{numberstyle=\ttfamily\color{Deeppurple}\upshape}
\lstset{numbersep=5pt}

\lstset{inputencoding=utf8/latin1}

%******************************************************************************************

\subsection{Salida de la simulación para la distorsión de salida con $R_{SS}$ desacoplada}

%*********************************************
\subsubsection{\quotefile{Distorsion\_wo\_Rss.out}}
\label{OUT_FILE_DIST_WO_RSS}
Simulación paramétrica en amplitud para la entrada del amplificador con $R_{SS}$ desacoplada.\\

\begin{flushleft}
\filebox{A continuación se muestra el archivo \textcolor{DarkBlue}{\quotefile{Distorsion\_wo\_Rss.out}}:}{Aluminium2}
\end{flushleft}

%\fontencoding{T1}
\fontseries{m}
\fontsize{5pt}{6pt}
\selectfont

\lstinputlisting{./sim_outs/Excursion_sin_Rss_m.out}

\normalfont
\normalsize
\selectfont

%*********************************************


\clearpage


%******************************************************************************************

\subsection{Salida de la simulación para la distorsión de salida con $R_{SS}$ sin desacoplar}

%*********************************************
\subsubsection{\quotefile{Distorsion\_w\_Rss.out}}
\label{OUT_FILE_DIST_W_RSS}
Simulación paramétrica en amplitud para la entrada del amplificador con $R_{SS}$ sin desacoplar.\\

\begin{flushleft}
\filebox{A continuación se muestra el archivo \textcolor{DarkBlue}{\quotefile{Distorsion\_w\_Rss.out}}:}{Aluminium2}
\end{flushleft}

%\fontencoding{T1}
\fontseries{m}
\fontsize{5pt}{6pt}
\selectfont

\lstinputlisting{./sim_outs/Excursion_con_Rss_m.out}

\normalfont
\normalsize
\selectfont

%*********************************************
