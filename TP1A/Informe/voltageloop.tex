
En esta sección se hace un análisis del comportamiento dinámico del amplificador del circuito a lazo abierto, por lo que se realiza en pequeña señal y a frecuencias bajas/medias.
El análisis se hace por inspección, pero teniendo en cuenta que se cumplan las condiciones que permite usar simplificaciones, de no ser así se usan expresiones mas completas para el cálculo. Dado que se usan valores posteriores en cada uno de los pasos, se hace el análisis usando nombres descriptivos y luego de calculados, se reemplazan estos valores.

\subsection{Análisis de la etapa diferencial, $Q_{12}/Q_{13}$}

Para la ganancia de tensión del diferencial, observamos que se trata de una etapa diferencial bipolar \textbf{NPN}, con carga activa, un espejo de corriente bipolar \textbf{PNP} con degeneración de emisor, con factor de copia igual a $1$, y cargado con la resistencia de entrada del seguidor que forma la segunda etapa.\\

\subsubsection{Ganancia de tensión}

Para la ganancia de esta etapa, tenemos:

\begin{equation}
A_{v_{D}} = gm_{D} \cdot \left( \left(  r_{d_{D_{1}}} + r_{o_{Q_{12}}} \right) \parallelresistors  R_{o_{mirror}} \parallelresistors R_{i_{seguidor_{Q_{3}}}} \right)
\end{equation}


\begin{equation}
gm_{D} \approx \frac{gm_{12} + gm_{13}}{2}
\end{equation}


\begin{equation*}
gm_{D} \approx \frac{\num{4.39E-2} \si[per-mode=symbol]{\ampere\per\volt} + \num{4.85E-2} \si[per-mode=symbol]{\ampere\per\volt}}{2} = \num{4.62e-2} \si[per-mode=symbol]{\ampere\per\volt}
\end{equation*}


\begin{equation*}
A_{v_{D}} = \num{4.62e-2} \si[per-mode=symbol]{\ampere\per\volt} \times \left( \left( 46.8 \si[per-mode=symbol]{\ohm}  + 56 \si[per-mode=symbol]{\kilo\ohm} \right) \parallelresistors 1.04 \si[per-mode=symbol]{\mega\ohm} \parallelresistors 1.76 \si[per-mode=symbol]{\mega\ohm}  \right) = 2384.89
\end{equation*}

\subsubsection{Resistencia de entrada}

\begin{equation}
R_{i_{D}} = 2 \cdot r_{\pi_{D}}
\end{equation}

\begin{equation}
r_{\pi_{D}} \approx \frac{r_{\pi_{12}} + r_{\pi_{13}}}{2}
\end{equation}

\begin{equation*}
r_{\pi_{D}} \approx \frac{7.21 \si[per-mode=symbol]{\kilo\ohm} + 8.12 \si[per-mode=symbol]{\kilo\ohm}}{2} = 7.67 \si[per-mode=symbol]{\kilo\ohm}
\end{equation*}

\begin{equation*}
R_{i_{D}} = 2 \times 7.67 \si[per-mode=symbol]{\kilo\ohm} = 15.34 \si[per-mode=symbol]{\kilo\ohm}
\end{equation*}



\subsubsection{Resistencia de salida}

\begin{equation}
R_{o_{D}} = r_{o_{12}} \parallelresistors R_{o_{mirror}}
\end{equation}


\begin{equation*}
R_{o_{D}} = 56 \si[per-mode=symbol]{\kilo\ohm} \parallelresistors 1.04 \si[per-mode=symbol]{\mega\ohm} = 53.14 \si[per-mode=symbol]{\kilo\ohm}
\end{equation*}


\subsubsection{Análisis de la fuente espejo - resistencia de salida}

La fuente espejo tiene degeneración por emisor, lo que constituye realimentación \textbf{serie-serie}, lo cual aumenta su resistencia de salida.\\ \\
Dado que en este caso se cumple la aproximación $gm \cdot R_{E} << \beta$, se tiene:

\begin{equation}
R_{o_{mirror}} = R_{o_{C_{Q_{1}}}} \approx \left( 1 + gm_{1} \cdot R_{2} \right) \cdot r_{o_{1}}
\end{equation} 

\begin{equation*}
R_{o_{mirror}} = R_{o_{C_{Q_{1}}}} \approx \left( 1 + \num{4.89E-2} \si[per-mode=symbol]{\ampere\per\volt}  \times 500 \si[per-mode=symbol]{\ohm}  \right) \times 40.7 \si[per-mode=symbol]{\kilo\ohm} = 1.04 \si[per-mode=symbol]{\mega\ohm}
\end{equation*} 

\subsection{Análisis de la etapa en seguidor por emisor, $Q_{3}$}


\subsubsection{Análisis de la ganancia de tensión}

Dado que este circuito no cumple la condición, $r_{o} \gg R_{E}$ no se puede usar la expresión aproximada $A_{v} \approx \frac{gm \cdot R_{E}}{1 + gm \cdot R_{E}}$, con lo que tenemos:

\begin{equation}
A_{v_{Q_{3}}} = \frac{1}{1 + \frac{r_{\pi_{3}}}{  \left(  \beta_{3} + 1 \right) \cdot \left(  R_{o_{C_{Q_{6}}}} \parallelresistors R{i_{Sziklai}} \parallelresistors r_{o_{3}}  \right)  } }
\end{equation}

\begin{equation*}
A_{v_{Q_{3}}} = \frac{1}{1 + \frac { 197 \si[per-mode=symbol]{\ohm} }{  \left(  130 + 1 \right) \times \left(  925.98 \si[per-mode=symbol]{\kilo\ohm} \parallelresistors 773.17 \si[per-mode=symbol]{\kilo\ohm} \parallelresistors 10.5 \si[per-mode=symbol]{\kilo\ohm}  \right)  } } = 0.9998
\end{equation*}

\subsubsection{Análisis de la resistencia de entrada}


Para calcular la  resistencia de entrada, vemos que no se cumple la condición $r_{o} \gg R_{E}$, con lo que la expresión aproximada, $ R_{i_{B_{Q_{3}}}} \approx r_{\pi_{3}} + \beta_{3} \cdot R_{E}$ no se puede usar.\\ \\

Tenemos entonces:

\begin{equation}
R_{i_{seguidor_{Q_{3}}}} = R_{i_{B_{Q_{3}}}} = r_{\pi_{3}} + \left( \beta_{3} + 1 \right) \cdot  \left(  R_{o_{C_{Q_{6}}}} \parallelresistors R{i_{Sziklai}}   \right)  \cdot \frac{  r_{o_{3}} + \frac{ R_{34} }{\left( \beta_{3} + 1 \right)} }{  r_{o_{3}} + R_{34} +  \left(  R_{o_{C_{Q_{6}}}} \parallelresistors R{i_{Sziklai}}   \right)  }
\end{equation}


\begin{equation*}
R_{i_{seguidor_{Q_{3}}}} = R_{i_{B_{Q_{3}}}} = 197 \si[per-mode=symbol]{\ohm} + \left( 130 + 1 \right) \times \left(  925.98 \si[per-mode=symbol]{\kilo\ohm} \parallelresistors 773.17 \si[per-mode=symbol]{\kilo\ohm} \right)  \times \frac{  13.9 \si[per-mode=symbol]{\kilo\ohm} + \frac{ 10 \si[per-mode=symbol]{\ohm} }{\left( 130 + 1 \right)} }{  13.9 \si[per-mode=symbol]{\kilo\ohm} + 10 \si[per-mode=symbol]{\ohm} +  \left(  925.98 \si[per-mode=symbol]{\kilo\ohm} \parallelresistors 773.17 \si[per-mode=symbol]{\kilo\ohm} \right)  } = 1.76 \si[per-mode=symbol]{\mega\ohm} 
\end{equation*}


\subsubsection{Análisis de la fuente de corriente, $Q_{6}$ - resistencia de salida}

Dado que en este caso no se cumple la aproximación $gm \cdot R_{E} << \beta$, se tiene:


\begin{equation}
R_{o_{C_{Q_{6}}}} = r_{o_{6}} \cdot \left( 1 + \frac{  gm_{6} \cdot R_{8}  }{  1 + \frac{ gm_{6} \cdot R_{8}  }{ \beta_{6} }   }     \right)
\end{equation}


\begin{equation*}
R_{o_{C_{Q_{6}}}} = 13 \si[per-mode=symbol]{\kilo\ohm} \times \left( 1 + \frac{  \num{6.62E-1} \si[per-mode=symbol]{\ampere\per\volt} \times 250 \si[per-mode=symbol]{\ohm}  }{  1 + \frac{  \num{6.62E-1} \si[per-mode=symbol]{\ampere\per\volt} \times 250 \si[per-mode=symbol]{\ohm}  }{ 122 }   }     \right) = 925.98 \si[per-mode=symbol]{\kilo\ohm}
\end{equation*}



\subsection{Análisis de la ganancia de la etapa del par compuesto (Sziklai), $Q_{4}/Q_{5}$}

Usamos las expresiones obtenidas en la sección~\sectref{section:sziklai}, adaptándolas a este caso particular.


\subsubsection{Ganancia de tensión}

\begin{equation}
a_{Sziklai} = gm_{4} \cdot \left( r_{0_{4}} \parallelresistors R_{11} \parallelresistors r_{\pi_{2}} \right) \cdot gm_{5} \cdot \left( r_{o_{5}} \parallelresistors R{L_{Sziklai}} \right)
\end{equation}

\begin{equation}
R{L_{Sziklai}} \approx R_{S} + R_{L} \parallelresistors \left( R_{9} + R_{10} \right)
\end{equation}

\begin{equation*}
R{L_{Sziklai}} \approx 0.2 \si[per-mode=symbol]{\ohm} + 100 \si[per-mode=symbol]{\ohm} \parallelresistors \left( 10 \si[per-mode=symbol]{\ohm} + 10 \si[per-mode=symbol]{\ohm} \right) = 99.7 \si[per-mode=symbol]{\ohm}
\end{equation*}


\begin{equation*}
a_{Sziklai} = \num{1.21E-1} \si[per-mode=symbol]{\ampere\per\volt} \times \left( 10.5 \si[per-mode=symbol]{\kilo\ohm} \parallelresistors 100 \si[per-mode=symbol]{\ohm} \parallelresistors  5.74 \si[per-mode=symbol]{\kilo\ohm} \right) \cdot \num{7.43E-1} \si[per-mode=symbol]{\ampere\per\volt} \times \left( 2.1 \si[per-mode=symbol]{\kilo\ohm} \parallelresistors 99.7 \si[per-mode=symbol]{\ohm} \right) = 833.06
\end{equation*}



\subsubsection{Resistencia de entrada}

\begin{equation}
R{i_{Sziklai}} = \left( 1 + a \cdot f \right) \cdot r_{\pi_{4}}
\end{equation}

\begin{equation*}
R{i_{Sziklai}} = \left( 1 + 833.06 \times 1 \right) \times 927  \si[per-mode=symbol]{\ohm} = 773.17 \si[per-mode=symbol]{\kilo\ohm}
\end{equation*}



\subsubsection{Resistencia de salida}

\begin{equation}
R{0_{Sziklai}} = \frac{ r_{o_{5}} \parallelresistors R{L_{Sziklai}}  }{1 + a \cdot f}
\end{equation}

\begin{equation*}
R{0_{Sziklai}} = \frac{ 2.1 \si[per-mode=symbol]{\kilo\ohm} \parallelresistors 99.7 \si[per-mode=symbol]{\ohm}  }{  1 + 833.06 \times 1 } = 114.11 \si[per-mode=symbol]{\milli\ohm}
\end{equation*}








