Se colocaron capacitores (\texttt{C4},\texttt{C5},\texttt{C6} y \texttt{C7}) en los 4 rieles de alimentación ($\pm 50V$ y $\pm 15V$) para filtrar ruidos en los rieles de alimentación.



\label{sec:fuente}

En nuestro proyecto como ya acordamos vamos a utilizar la fuente conmutada realizada por el grupo Arias-Caracciolo-Luna.
A continuación se detallan las especificaciones basicas de la misma:

\begin{itemize}
\item{Tensión nominal de entrada: 220 V AC rms}
\item{Rango de tension de entrada: 176 V AC rms − 242 V AC rms}
\item{Rango de frecuencia de entrada: 45 Hz − 60 Hz}
\item{Tension nominal de salida:$ V_{OUT} = ± 15 V$ , $V_{OUT} = ± 50 V$}
\item{Rango de corriente de carga: Para ± 50 V : 2,5 A , Para ± 15 V : 0,7 A , Minima corriente de carga: 0 A}
\item{Potencia de entrada sin carga: Menor a 5 W para todo el rango de tensión de entrada.}
\item{Potencia de salida: Para el limite superior de la corriente de carga es 125 W}
\item{Rendimiento: Mayor a $70\%$}
\item{Protecciones de sobretensión a ± 60 V (para las salidas de ± 50 V ) y a ± 20 V (paralas salidas de ± 15 V )}
\item{Protección de cortocircuito: Clamp.}
\item{Factor de rizado (ripple): Para todas las salidas es menor a 300 mV p−p}
\item{Estabilidad a largo plazo: Tensión de ruido a la salida: Menor al $5\%$}
\item{Tipo/s de conector/es de salida: Banana hembra de 4 mm  (tipo RC170). Fijados al chasis de montaje}
\end{itemize}
