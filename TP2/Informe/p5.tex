\vspace{1.5cm}
\label{calculated_zi}

Para calcular la impedancia de entrada a lazo cerrado, debemos primero calcular la impedancia de entrada a lazo abierto, tenemos:

\begin{align}
\begin{split}
R_{i_{OL}} &= R_{i_{B_{402}}} \approx r_{\pi_{402}} + R_{E_{1}} \cdot \beta_{402} = \\
&= r_{\pi_{402}} + \left( R_{410} \parallelresistors R_{411} \right) \cdot \beta_{402} = \\
&= 15.3 \si[per-mode=symbol]{\kilo\ohm} + \left( 120 \si[per-mode=symbol]{\ohm} \parallelresistors 3.3 \si[per-mode=symbol]{\kilo\ohm} \right) \cdot 330 = 53.5 \si[per-mode=symbol]{\kilo\ohm}
\end{split}
\end{align}

Luego por tratarse de un circuito realimentado que suma tensión, tenemos

\begin{align}
\begin{split}
R_{i} &= \left[\left(1 + a \cdot f \right) \cdot R_{i_{OL}} \right] \parallelresistors R_{B} = \\
&= \left[\left(1 + a \cdot f \right) \cdot R_{i_{OL}} \right] \parallelresistors \left[ \left( R_{5} + PS_{401} \right) \parallelresistors R_{407} \right] = \\
&= \left[\left(1 + 6718 \cdot 0.035 \right) \cdot 53.5 \si[per-mode=symbol]{\kilo\ohm} \right] \parallelresistors \left[ \left( 150 \si[per-mode=symbol]{\kilo\ohm} + 50.9 \si[per-mode=symbol]{\kilo\ohm} \right) \parallelresistors 150 \si[per-mode=symbol]{\kilo\ohm} \right] \approx 85.3 \si[per-mode=symbol]{\kilo\ohm}
\end{split}
\end{align}

\begin{equation}
\boxed{ R_{i} \approx 85.3 \si[per-mode=symbol]{\kilo\ohm} }
\end{equation}



\vfill

\clearpage
