\label{qpoint}

\begin{sloppypar}

Para el cálculo manual del punto de reposo, se procedió a tomar en cuenta primero los requisitos de salida, $V_{o_{Q}} = 0 \si[per-mode=symbol]{\volt}$, $I_{C_{Q_{407}}}=10 \si[per-mode=symbol]{\milli\ampere}$ y $I_{C_{Q_{408}}}=10 \si[per-mode=symbol]{\milli\ampere}$, se despreciaron además todas las corrientes de base de los transistores (se verificó la asunción luego de cada cálculo), excepto $I_{B_{Q_{402}}}$, para ajustar $PS_{401}$. \\
Todas los $V_{BE}$ de los transistores se tomaron de las curvas dadas en las hojas de datos correspondientes, una vez obtenida la corriente de colector, para tener algo mas de precisión en los cálculos.\\


En detalle, se realizaron los siguientes pasos (los mismos están marcados y se pueden seguir en la \mbox{figura~\figref{fig:fig_calculated_qpoint}}) para hallar el punto de reposo:



\begin{enumerate}

\item[\textbf{(1)}] $V_{B_{Q_{407}}} =  V_{o_{Q}} + V_{BE_{Q_{407}}}  = 0 \si[per-mode=symbol]{\volt} + 0.6 \si[per-mode=symbol]{\volt} = 0.6 \si[per-mode=symbol]{\volt}$, se desprecia la caída en $R_{421}$ y el $V_{BE}$ es tomado de la \textbf{Figura~10},\, \quotemarks{On~Voltages}, de la hoja de datos del transistor \textbf{TIP41}, apéndice~\sectref{datasheet_TIP41}.


\item[\textbf{(1')}] Despreciando la caída en $R_{421}$, se tiene $I_{R_{419}}=\dfrac{V_{BE_{Q_{407}}}}{R_{419}}=  \frac{0.6 \si[per-mode=symbol]{\volt}}{100 \si[per-mode=symbol]{\ohm}}  =6\si[per-mode=symbol]{\milli\ampere}$.


\item[\textbf{(2)}]  $V_{B_{Q_{405}}}=1.2 \si[per-mode=symbol]{\volt}$ , a $V_{B_{Q_{407}}}$ se suma $V_{BE_{Q_{405}}}=0.6 \si[per-mode=symbol]{\volt}$, este último tomado de la \textbf{Figura~4}, \quotemarks{\mbox{Base-Emitter~On~Voltage}}, de la hoja de datos del transistor \textbf{BD135}, apéndice~\sectref{datasheet_BD135}, tomando la corriente obtenida en \textbf{(1')}.


\item[\textbf{(3)}] $I_{C_{Q_{404}}} = \frac{V_{CC} - V_{B_{Q_{405}}}}{R_{413} + R_{414}} = \frac{30 \si[per-mode=symbol]{\volt} - 1.2 \si[per-mode=symbol]{\volt}}{1 \si[per-mode=symbol]{\kilo\ohm + 2.2 \si[per-mode=symbol]{\kilo\ohm}}} = 9 \si[per-mode=symbol]{\milli\ampere}$.


\item[\textbf{(4)}] $I_{C_{Q_{402}}}=\frac{V_{BE_{Q_{404}}}}{R_{408}} =  \frac{0.65 \si[per-mode=symbol]{\volt}}{1.2 \si[per-mode=symbol]{\kilo\ohm}}  = 0.54 \si[per-mode=symbol]{\milli\ampere}$ , donde $V_{BE_{Q_{404}}}$ se toma igual que antes de las hoja de datos, usando la corriente obtenida en \textbf{(3)}.


\item[\textbf{(5)}] $V_{E_{Q_{402}}}=V_{o_{Q}}-I_{C_{Q_{402}}} \cdot R_{411} = 0 \si[per-mode=symbol]{\volt} - 0.54 \si[per-mode=symbol]{\milli\ampere} \cdot 3.3 \si[per-mode=symbol]{\kilo\ohm} \approx -1.8 \si[per-mode=symbol]{\volt}$.


\item[\textbf{(5')}] $V_{B_{Q_{402}}} = V_{E_{Q_{402}}} + V_{BE_{Q_{402}}} = -1.8 \si[per-mode=symbol]{\volt} - 0.65 \si[per-mode=symbol]{\volt} = -2.45 \si[per-mode=symbol]{\volt}$ , donde $V_{BEQ402}$ se toma de la \textbf{Figura~2}, \quotemarks{Saturation~and~On~Voltages}, de la hoja de datos del transistor \textbf{BC558}, apéndice~\sectref{datasheet_BC558}, usando la corriente obtenida en \textbf{(4)}.

\item[\textbf{(6)}] $I_{R_{422}} = I_{C_{Q_{407}}}+I_{R_{419}}-I_{R_{411}} = 10 \si[per-mode=symbol]{\milli\ampere} + 6 \si[per-mode=symbol]{\milli\ampere} - 0.5 \si[per-mode=symbol]{\milli\ampere} = 15.5 \si[per-mode=symbol]{\milli\ampere} $.

\item[\textbf{(7)}] $V_{B_{Q_{408}}} = V_{SS} + V_{BE_{Q_{408}}} = -30 \si[per-mode=symbol]{\volt} + 0.55 \si[per-mode=symbol]{\volt} = -29.45 \si[per-mode=symbol]{\volt} $, el $V_{BE}$ se toma igual que antes de las hoja de datos, se elige menor que en \textbf{(1)} debido a que en este caso no tenemos resistencia de emisor y además coincida con $I_{C_{Q_{406}}}$, que hallamos en el siguiente item.


\item[\textbf{(7')}] $I_{C_{Q_{406}}} = 15.5 \si[per-mode=symbol]{\milli\ampere} - 10 \si[per-mode=symbol]{\milli\ampere} = 5.5 \si[per-mode=symbol]{\milli\ampere} $ , $V_{B_{Q_{406}}} = -0.6 \si[per-mode=symbol]{\volt} $ , calculado despreciando la caída en $R_{422}$ a partir de $V_{BE}$ tomado de la \textbf{Figura~4}, \quotemarks{\mbox{Base-Emitter~On~Voltage}}, de la hoja de datos del transistor \textbf{BD136}, apéndice~\sectref{datasheet_BD136}, usando $I_{C_{Q_{406}}}$, recién obtenida.


\item[\textbf{(8)}] $V_{CE_{Q_{403}}} = V_{C_{Q_{403}}} - V_{E_{Q_{403}}} = 1.2 \si[per-mode=symbol]{\volt} - (-0.6 \si[per-mode=symbol]{\volt}) = 1,8 \si[per-mode=symbol]{\volt} $.


\item[\textbf{(9)}] $I_{R_{415}} = \frac{V_{CE_{Q_{403}}}}{R_{415} + PS_{402} + R_{416} } = \frac{1.8 \si[per-mode=symbol]{\volt} }{2.2 \si[per-mode=symbol]{\kilo\ohm} + 2.2 \si[per-mode=symbol]{\kilo\ohm} + 1 \si[per-mode=symbol]{\kilo\ohm} } = 0.33 \si[per-mode=symbol]{\milli\ampere} $.


\item[\textbf{(9')}] $I_{C_{Q_{403}}} = I_{R_{414}} - I_{R_{415}} = 9 \si[per-mode=symbol]{\milli\ampere} - 0.33 \si[per-mode=symbol]{\milli\ampere} = 8.67 \si[per-mode=symbol]{\milli\ampere} $.


\item[\textbf{(10)}] Primeramente obtenemos $V_{BE_{Q_{403}}} = 0.7 \si[per-mode=symbol]{\volt} $, se toma de la \textbf{Figura~2}, \quotemarks{Saturation~and~On~Voltages}, de la hoja de datos del transistor \textbf{BC548}, apéndice~\sectref{datasheet_BC548}.\\

Luego tenemos que $PS_{402_{B}} = \frac{V_{BE_{Q_{403}}}}{I_{R_{415}}} - R_{416} = \frac{0.7 \si[per-mode=symbol]{\volt} }{0.33 \si[per-mode=symbol]{\milli\ampere}  } - 1 \si[per-mode=symbol]{\kilo\ohm} = 1121 \si[per-mode=symbol]{\ohm} $, con lo que $PS_{402_{A}} = 2200 \si[per-mode=symbol]{\ohm} - 1121 \Omega = 1079 \si[per-mode=symbol]{\ohm} $\\

Finalmente:
\begin{center}
$PS_{402_{A}} = 1079 \si[per-mode=symbol]{\ohm} $  y  $PS_{402_{B}} = 1121 \si[per-mode=symbol]{\ohm} $
\end{center}


\item[\textbf{(11)}] Primero se considera que $I_{R_{418}} = 9.54 \si[per-mode=symbol]{\milli\ampere} $, la suma de $I_{C_{Q_{404}}}$ e $I_{R_{408}}$, circulan por $R_{418}$, con esto se obtuvo un primer valor de $V_{E_{Q_{404}}} = -28.57 \si[per-mode=symbol]{\volt} $, primera aproximación (se calcula en el punto \textbf{(12)}), con esto último se obtiene $I_{R_{407}} = \frac{V_{B_{Q_{402}}} - V_{E_{Q_{404}}} }{R_{407} + R_{10} } = \frac{-2.45 \si[per-mode=symbol]{\volt} - (-28,57 \si[per-mode=symbol]{\volt})}{197 \si[per-mode=symbol]{\kilo\ohm} } = 132.6 \si[per-mode=symbol]{\micro\ampere} $, con esto obtenemos el valor final de $I_{R_{418}} = 9.67 \si[per-mode=symbol]{\milli\ampere} $.


\item[\textbf{(12)}] $V_{E_{Q_{404}}} = I_{R_{418}} \cdot R_{418} + V_{SS}  = 9.67 \si[per-mode=symbol]{\milli\ampere} \cdot 150 \si[per-mode=symbol]{\ohm} - 30 \si[per-mode=symbol]{\volt} = -28.55 \si[per-mode=symbol]{\volt} $.



\item[\textbf{(13)}] Con este valor reajustamos $I_{R_{407}}$ para ajustar mejor $PS_{401}$, quedando finalmente  

$I_{R_{407}} =\frac{V_{B_{Q_{402}}} - V_{E_{Q_{404}}} }{R_{407} + R_{10}}   = \frac{-2.45 \si[per-mode=symbol]{\volt} - (-28.55 \si[per-mode=symbol]{\volt})}{197 \si[per-mode=symbol]{\kilo\ohm}} = 132.5 \si[per-mode=symbol]{\micro\ampere} $.


\item[\textbf{(14)}] $I_{B_{Q_{402}}} = \frac{ I_{C_{Q_{402}}} }{ \beta_{402} } = \frac{ 0.54 \si[per-mode=symbol]{\milli\ampere} }{ 330 } = 1.6 \si[per-mode=symbol]{\micro\ampere}  $.


\item[\textbf{(14')}] $I_{PS_{401}} = I_{R_{407}} - I_{B_{Q_{402}}} = 130.9 \si[per-mode=symbol]{\micro\ampere} $.


\item[\textbf{(15)}] $PS_{401} = \frac{ V_{CC} - V_{B_{Q_{402}}} }{I_{PS_{401}} } - R_{5} - R_{405} =  \frac{ 30 \si[per-mode=symbol]{\volt} - (-2,45 \si[per-mode=symbol]{\volt} )}{130.9 \si[per-mode=symbol]{\micro\ampere} } - 150 \si[per-mode=symbol]{\kilo\ohm} - 47 \si[per-mode=symbol]{\kilo\ohm} \approx 50.9 \si[per-mode=symbol]{\kilo\ohm} $\\

\end{enumerate}

\end{sloppypar}

Se verifica que todos los transistores están en modo activo directo. El punto de reposo obtenido se muestra sobre el circuito en la figura~\figref{fig:fig_calculated_qpoint}, las corrientes de colector y los elementos del modelo de pequeña señal para los transistores se resumen en en el cuadro~\tableref{table:table_qpoint}.


%% \noindent
%% \begin{center}
 
%%\begin{spacing}{1}  
\begin{table}[H]  %%\centering
    
    \setlength\arrayrulewidth{1.5pt}
    \arrayrulecolor{white}
    \def\clinecolor{\hhline{|>{\arrayrulecolor{white}}-%
    >{\arrayrulecolor{white}}|-|-|-|-|-|-|-|-|}}
\resizebox{0.9 \textwidth}{!}{% 
       
\begin{tabularx}{1 \textwidth}%
    {|
    >{\columncolor{white} \centering\arraybackslash}m{0.130\linewidth}
     |
    >{\columncolor{white} \centering\arraybackslash}m{0.065\linewidth}
     |
    >{\columncolor{white} \centering\arraybackslash}m{0.065\linewidth}
     |
    >{\columncolor{white} \centering\arraybackslash}m{0.065\linewidth}
     |
    >{\columncolor{white} \centering\arraybackslash}m{0.065\linewidth}
     |
    >{\columncolor{white} \centering\arraybackslash}m{0.065\linewidth} 
     |
    >{\columncolor{white} \centering\arraybackslash}m{0.065\linewidth}  
     |
    >{\columncolor{white} \centering\arraybackslash}m{0.065\linewidth}  
     |
    >{\columncolor{white} \centering\arraybackslash}m{0.065\linewidth} 
     |
    >{\columncolor{white} \centering\arraybackslash}m{0.065\linewidth}  
     |
    >{\columncolor{white} \centering\arraybackslash}m{0.065\linewidth} 
     |
    >{\columncolor{white} \centering\arraybackslash}m{0.065\linewidth} 
     |
    }
    \rowcolor{HeadersColor} \cellcolor{white} \thead{}  & \thead{$Q_{402}$} & \thead{$Q_{403}$} & \thead{$Q_{404}$} & \thead{$Q_{405}$} & \thead{$Q_{406}$} & \thead{$Q_{407}$} & \thead{$Q_{408}$} \\
    
    \hhline{|-|-|-|-|-|-|-|-|}
    \rowcolor{Butter!20} \cellcolor{Butter!40} $I_{C}$ [$\si[per-mode=symbol]{\milli\ampere}$] & $0.54$ & $8.66$ & $9$ & $6$ & $5.5$ & $10$ & $10$  \\
    \hhline{|-|-|-|-|-|-|-|-|}
    \rowcolor{gray!20} \cellcolor{gray!40} $gm$ [$\si[per-mode=symbol]{\milli\ampere\per\volt}$] & $21.6$ & $346$ & $360$ & $240$ & $220$ & $400$ & $400$  \\
    \hhline{|-|-|-|-|-|-|-|-|}
    \rowcolor{gray!20} \cellcolor{gray!40}  $r_{\pi}$ [$\si[per-mode=symbol]{\ohm}$] & $15.3 \si[per-mode=symbol]{\kilo}$ & $953.8$ & $388.9$ & $583.3$ & $636.4$ & $125$ & $125$  \\
    \hhline{|-|-|-|-|-|-|-|-|}
    \rowcolor{gray!20} \cellcolor{gray!40} $r_{o}^{\left(*\right)}$ [$\si[per-mode=symbol]{\kilo\ohm}$] & $128$ & $7.43$ & $15.9$ & $25.9$ & $22.4$ & $12.8$ & $12.9$  \\
    \hhline{|-|-|-|-|-|-|-|-|}
    \rowcolor{gray!20} \cellcolor{gray!40} $\beta^{\left(**\right)}$ & $330$ & $330$ & $140$ & $210$ & $200$ & $50$ & $50$  \\
    \hhline{|-|-|-|-|-|-|-|-|}          
    \end{tabularx}}
	\caption{\footnotesize{Elementos del modelo de pequeña señal de los transistores ($f_{\left(I_{C}\right)}$).}}
	\label{table:table_qpoint}
\end{table}
%%\end{spacing}

%% \end{center}

\footnotesize{
$\left(*\right)$ Las tensiones de Early ($V_{A}$) de los transistores usadas para el cálculo de los $r_{o}$ fueron obtenidas de las simulaciones, ya que no todas las hojas de datos proveían el dato.\\
$\left(**\right)$ Los $\beta$ de los transistores fueron obtenidos de las hojas de datos, usando las curvas cuando estas se proveían, o valores aproximados en otros casos, de acuerdo al punto de trabajo.
}


\vfill









