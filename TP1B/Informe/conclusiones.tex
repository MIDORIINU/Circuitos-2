
Cuando empezamos la elaboración del trabajo práctico intentamos desarrollarlo como si diseñáramos la compensación desde cero, a pesar de que las redes que concluimos que serían necesarias coincidían en la localización con las existentes en el diseño, nos encontramos que el cálculo de los valores requería un desarrollo teórico extenso y una validación iterativa por simulación que se hacia muy complicada en el tiempo disponible, no obstante este primer intento nos sirvió para entender que involucra el diseño de una compensación. \\
Otra cosa en el desarrollo del análisis por simulación es que nos costó un poco en algunos casos ver porque un valor es mejor que otro, al menos por lo que se podía ver en las simulaciones, eso puede ser porque faltan casos que simular, pero ya era bastante extenso como para hacer mas simulaciones. \\
Finalmente, viendo lo justa que queda la compensación para algunos casos, llegamos a la conclusión que sería necesario un análisis por el \textbf{método de Monte Carlo} para garantizar la estabilidad frente a las variaciones de los valores de los componentes, pero nuevamente, el análisis sería demasiado extenso, especialmente dado el tiempo que una simulación estadística puede tomar.\\
Por último, queda claro que la compensación que el circuito tiene es en general suficiente para garantizar la estabilidad de la fuente de alimentación.\\